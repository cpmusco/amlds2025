\documentclass[11pt]{article}
\usepackage{fullpage}
\usepackage{amsmath,amsfonts,amsthm,amssymb}
\usepackage{url}
\usepackage[demo]{graphicx}
\usepackage{caption} 
\usepackage{algpseudocode}
\usepackage{bbm}
\usepackage{float}
\usepackage{framed}
\usepackage{enumerate}
\usepackage{color}
\usepackage[colorlinks=true, linkcolor=red, urlcolor=blue, citecolor=blue]{hyperref}

\DeclareMathOperator*{\E}{\mathbb{E}}
\DeclareMathOperator*{\R}{\mathbb{R}}
\DeclareMathOperator{\var}{Var}

\topmargin 0pt
\advance \topmargin by -\headheight
\advance \topmargin by -\headsep
\textheight 8.9in
\oddsidemargin 0pt
\evensidemargin \oddsidemargin
\marginparwidth 0.5in
\textwidth 6.5in

\parindent 0in
\parskip 1.5ex

\newcommand{\homework}[2]{
	\noindent
	\begin{center}
		\framebox{
			\vbox{
				\hbox to 6.50in { {\bf NYU CS-GY 6763: Algorithmic Machine Learning 
						and Data Science} \hfill Spring 2025 }
				\vspace{4mm}
				\hbox to 6.50in { {\Large \hfill Homework #2  \hfill} }
				\vspace{2mm}
				\hbox to 6.50in { {Name: #2 \hfill} }
			}
		}
	\end{center}
	\vspace*{4mm}
}

\begin{document}
	
	\homework{1}{Solution key}

	\section*{Problem 1}
\smallskip\noindent 1.\hspace{1em} $\E[X] = 1/2$, so by Markov's $\Pr[X \geq 7/8]\leq \frac{1/2}{7/8} = .571$. 

\smallskip\noindent 2.\hspace{1em} $\E[(X-\E X)^2] = \int_{0^1} (x-.5)^2 dx = .08333\ldots$ via Wolfram Alpha. So by Chebyshev's $\Pr[|X - .5| \geq (7/8-.5)]\leq \frac{.08333}{(7/8-.5)^2} = .593$. 

\smallskip\noindent 3.\hspace{1em} $\E[X^2] = \int_{0}^1 x^2 dx = \frac{1}{3}$, so by Markov's $\Pr[X^2 \geq (7/8)^2]\leq \frac{1/3}{(7/8)^2} = .435$. So the uncentered moment gives a better bound.

\smallskip\noindent 4.\hspace{1em} The general equation for an upper bound is:
\begin{align*}
	\frac{1/(q+1)}{(7/8)^q}. 
\end{align*}
For $q = 3, 4, \ldots, 10$ we get values:
\begin{align*}
	\begin{bmatrix}.373 & .341 & .325 &	.318& .318& .323&.333&.346
	\end{bmatrix}
\end{align*}
So using higher moments at first improves our bound, but then eventually starts giving a weaker bound. The tighted bound is obtained at $q = 6$ and $q=6$. 

\smallskip\noindent 5.\hspace{1em} Let $g$ be the step function which is $0$ for $X< 7/8$ and $1$ for $X \geq 7/8$. Then we have $\Pr[X \geq 7/8] = \Pr[g(X) \geq 1]$. We have $\E[g(X)] = \frac{1}{8}$, so $\Pr[g(X) \geq 1] \leq \frac{1}{8}$ by Markov's, which gives the tight bound. 

\section*{Problem 2}
I will write this later. This is the standard ``median trick''. The difficult observation is that, for the median to give error $> \epsilon$, it must be that more than half of the runs of the algorithm give error $> \epsilon$. Why? Suppose the median gave an answer more than $\epsilon$ above the correct value. Then all values above the median (which is $1/2$ of the runs) have to also be $>\epsilon$ above the correct value. 

Once you make this observation, they just need to prove that if you flip a coin that comes up heads with probability $2/3$, you won't have $<50\%$ heads in a sample of size $O(\log(1/\delta))$ with probability $1-\delta$. This was actually done as an example in Lecture 3 (not with exactly those numbers). Best way is to use Chernoff bound.

\section*{Problem 3}
	\smallskip\noindent 1.\hspace{1em} 
	Once there are $n+1$ servers in this setup, the expected number of items on the $(n+1)^\text{st}$ server is $\frac{m}{n+1}$, by symmetry. All of these items (and only these items) must have been relocated when the $(n+1)^{\text{st}}$ server was added. So the expected number of items that move is $\frac{m}{n+1}$. 
	
	\smallskip\noindent 2. \hspace{1em} For a server $S$ to own more than a $c\log n/n$ fraction of the interval , it would need to be that \emph{no other server} falls within distance $c\log n/n$ to the left of the server. We can choose the random location of server $S$ first. Then the probability of any one server landing within distance $c\log n /n$ from $S$'s left is $c\log n / n$. So the probability \emph{no servers} land that close is:
	\begin{align*}
		(1 - c\log n / n)^{n-1} \leq \frac{1}{10n},
	\end{align*}
	as long as we choose $c$ to be a large enough constant (same analysis as homework 1). By a \emph{union bound}, we thus have that no server owns more than an $O(\log n / n)$ fraction of the interval with probability $\geq 1 - n\frac{1}{10 n} = \frac{9}{10}$ which proves the claim.
	
	\smallskip\noindent 3. \hspace{1em}
	From Part 2, we could have equivalently proven that no server owns more than a $c\log n / n$ fraction of the interval with probability $19/20$ (by choosing $c$ larger). For the rest of the problem, assume that this event happening.
	
	For servers $S_1, \ldots, S_n$ let $Y_i^{(j)}$ be the indicator random variable that item $j$ lands within distance $c\log n / n$ to $S_i$'s left. 
	Let $X_i$ equal $X_i = \sum_{j=1}^m Y_i^{(j)}$.
	Since we assumed that no server owns more than a $c\log n / n$ fraction of the interval, $X_i$ is an \emph{upper bound} on the number of items assigned to server $i$. So it suffices to show that $X_i$ is not too large for all $i$. 
	
	To do so, note that, for a fixed $i$,  $Y_i^{(1)}, Y_i^{(2)}, \ldots, Y_i^{(m)}$ are an independent $\{0,1\}$ random variables, where each is $1$ with probability exactly $c\log n / n$. So they are just biased coin flips!
	
	Let $c > 2$ be a sufficiently large constant. Using the Chernoff bound from class with $\epsilon = c$, we get that:
	\begin{align*}
		\Pr[X_i \geq 2c \cdot \frac{m\log n}{n}] \leq e^{\frac{-c^2m\log n / n}{2+c}} \leq e^{\frac{-c\log n}{2}} \leq \frac{1}{20n}, 
	\end{align*}
	for large enough $c$. The last inequality uses that $m > n$ (as specified in the problem).
	
	We conclude via a union bound that no server is assigned more than $O(m\log n /n)$ items with probability $\frac{19}{20}$. 
	
	There's one last step -- we needed two events to hold for our proof to go through: 1) no server owns more than a $c\log n / n$ fraction of the interval and 2) no server was assigned two many items. Since each holds with probability $19/20$, by another union bound, both hold with probability $9/10$.
	

	
\section*{Problem 4 (a)}

1. \textbf{Expectation Calculation.} As in class, we have that $\E[\|\Pi x\|_2^2] =  \E[\langle \pi, x\rangle^2]$, where $\pi$ is a single unscaled row from the matrix $\Pi$. I.e. $\pi$ has length $n$ and contains random $\pm 1$ entries. We have:
\begin{align*}
	\E[\langle \pi, x\rangle^2] = \E\left[\left(\sum_{j=1}^n \pi_j x_j \right)^2\right] &= \E\left[\sum_{j=1}^n \pi_j^2 x_j^2\right] + \E\left[\sum_{i\neq j}^n \pi_i\pi_j x_jx_i \right]\\
	&= \sum_{j=1}^n \E\left[\pi_j^2 \right] x_j^2 +\sum_{i\neq j}^n  \E\left[\pi_i\pi_j\right] x_jx_i .
\end{align*}
The last equality follows from linearity of expectation. Since $\pi_i$ is independent of $\pi_j$, we have that for $j\neq i$, $\E\left[\pi_i\pi_j\right] = \E[\pi_i]\E[\pi_j] = 0$. On the other hand $\pi_j^2 = 1$ deterministically, so we have $\E\left[\pi_j^2 \right]  = 1$. Plugging in above, we find that 
\begin{align*}
	\E[\langle \pi, x\rangle^2] = \sum_{j=1}^n x_j^2 +\sum_{i\neq j}^n  0\cdot x_jx_i  =  \sum_{j=1}^n x_j^2 = \|x\|_2^2,
\end{align*}
as desired.

 \textbf{Variance Calculation.} Since $\|\Pi x\|_2^2 = \frac{1}{k}\sum_{i=1}^k \langle \pi^i, x\rangle^2$, where $\pi^1, \ldots, \pi^k$ are the unscaled rows of $\Pi$, we first observe that $\var[\|\Pi x\|_2^2 ] = \frac{1}{k}\var[\langle \pi, x\rangle^2]$ for a single random $\pm 1$ vector $\pi$. So we just need to bound $\var[\langle \pi_i, x\rangle^2]$. This gets a bit tricky! There are many ways to do it, but I think the easiest way is to take advantage of linearity of variance by writing:
 \begin{align*}
 	\langle \pi, x\rangle^2 = \sum_{j=1}^n \pi_j^2 x_j^2 + 2\sum_{i> j} \pi_i\pi_j x_ix_j. 
 \end{align*}
The terms in the first part of the sum are actually deterministic, since $\pi_j = 1$. The terms in the second part of the sum are random, but they are \emph{pairwise independent} since $\pi_i\pi_j$ is random $\pm 1$ and independent from any $\pi_i\pi_k$, $\pi_k\pi_j$, or $\pi_k\pi_{\ell}$. They are not mutually independent, but we only need pairwise independence to apply linearity of variance. 
Note that to make this claim it's important that I used the form $2\sum_{i> j}$ instead of $\sum_{i\neq j}$. If I did the later, there would be repeated random variables in the sum ($\pi_i\pi_j x_ix_j$  and $\pi_j\pi_i x_jx_i$). Writing the other way removes duplicates.
\begin{align*}
	\var[\langle \pi, x\rangle^2] = \sum_{j=1}^n \var[\pi_j^2 x_j^2] + 4\sum_{i> j} \var[\pi_i\pi_j x_jx_i] = 0 + 4\sum_{i> j} x_j^2x_i^2.
\end{align*}
Then finally we observe that:
\begin{align*}
	\|x\|_2^4 = \|x\|_2^2\cdot \|x\|_2^2 = (x_1^2 + \ldots + x_n^2)\cdot (x_1^2 + \ldots + x_n^2) \geq 2\sum_{i> j} x_j^2x_i^2.
\end{align*} 
Putting this together we have that $\var[\langle \pi, x\rangle^2]  \leq 2 	\|x\|_2^4$ and the result follows since $\var[\|\Pi x\|_2^2 ] = \frac{1}{k}\var[\langle \pi, x\rangle^2]$ as claimed above.

\vspace{.5em}
2. This just follows directly from Chebyshev's.

\vspace{.5em}
3. It's almost the same analysis as in part 1. The first thing to observe is that:
\begin{align*}
	\langle \Pi x, \Pi y\rangle  = \frac{1}{k}\sum_{i=1}^k \langle \pi^i, x\rangle \langle \pi^i, y\rangle. 
\end{align*}
So we have that $\E[\langle \Pi x, \Pi y\rangle] = \E[\langle \pi, x\rangle \langle \pi, y\rangle]$ and $\var[\langle \Pi x, \Pi y\rangle] = \frac{1}{k}\var[\langle \pi, x\rangle \langle \pi, y\rangle]$, where $\pi$ is a single random $\pm 1$ vector. We also have that 
\begin{align*}
	\langle \pi, x\rangle \langle \pi, y\rangle = \left(\sum_{j=1}^n \pi_j x_j \right)\cdot  \left(\sum_{j=1}^n \pi_j y_j \right) = \sum_{i=1}^n\pi_i^2 x_iy_i + \sum_{j\neq i}\pi_i\pi_j x_iy_j.
\end{align*}
From this it's clear that 
\begin{align*}
		\E[\langle \Pi x, \Pi y\rangle]  = \E[\langle \pi, x\rangle \langle \pi, y\rangle] = \sum_{i=1}^n x_iy_i  = \langle x,y\rangle, 
\end{align*}
as desired. 

The variance calculation is also a bit tricky since we need to make sure our sums involve pairwise independent random variables. We have that: 
\begin{align*}
	\langle \pi, x\rangle \langle \pi, y\rangle = \sum_{i=1}^n\pi_i^2 x_iy_i + \sum_{j> i}\pi_i\pi_j (x_iy_j + x_jy_i).
\end{align*}
Applying linearity of variance, we find that 
\begin{align*}
	\var[\langle \pi, x\rangle \langle \pi, y\rangle] = \sum_{j> i} (x_iy_j + x_jy_i)^2 &= \sum_{j> i} x_i^2y_j^2 + x_j^2y_i^2  + 2 x_ix_jy_iy_j \\
	&\leq 2 \sum_{j> i} x_i^2y_j^2 + x_j^2y_i^2 \\
	& \leq 2(x_1^2 + \ldots + x_n^2)(y_1^2 + \ldots + y_n^2) \\
	&= 2 \|x\|_2^2 \|y\|_2^2.
\end{align*}
In second to last inequality we have used that for any $a,b$, $2ab \leq a^2 + b^2$, which follows from the fact that $(a-b)^2 \geq 0$ for all $a,b$ (this is technically called the AM-GM inequality).

Overall, we get a variance bound of:
\begin{align*}
	\var[\langle \Pi x, \Pi y\rangle] \leq \frac{2}{k}\|x\|_2^2 \|y\|_2^2.
\end{align*}

Once they get the mean and variance, the bound just follows from applying Chebyshev inequality again.
	.
	\section*{Problem 4 (b)}
	1. Construct 2 length $U$ binary vectors $x$ and $y$ where $x_i = 1$ if $i \in X$ and $0$ otherwise, and $y_i = 1$ if $i \in Y$ and $0$ otherwise. Note that $|X\cap Y|$ is exactly equal to $\langle x, y\rangle$, so we can estimate the quantity using sketches $\Pi x$ and $\Pi y$. If we set $k = O(1/\epsilon^2)$, then with $9/10$ probability we will have:
	\begin{align*}
		\left|\langle  x, y \rangle  - \langle \Pi x, \Pi y \rangle \right| \leq \epsilon \|x\|_2 \|y\|_2
	\end{align*}
Note that $\|x\|_2^2 = |X|$ and $\|y\|_2^2 = |Y|$, which yields the bound. 

\end{document}