\documentclass[11pt]{article}
\usepackage{fullpage}
\usepackage{amsmath,amsfonts,amsthm,amssymb}
\usepackage{url}
\usepackage[demo]{graphicx}
\usepackage{caption} 
\usepackage{algpseudocode}
\usepackage{bbm}
\usepackage{float}
\usepackage{framed}
\usepackage{enumerate}
\usepackage{color}
\usepackage[colorlinks=true, linkcolor=red, urlcolor=blue, citecolor=blue]{hyperref}

\DeclareMathOperator*{\E}{\mathbb{E}}
\DeclareMathOperator*{\R}{\mathbb{R}}
\DeclareMathOperator{\var}{Var}

\topmargin 0pt
\advance \topmargin by -\headheight
\advance \topmargin by -\headsep
\textheight 8.9in
\oddsidemargin 0pt
\evensidemargin \oddsidemargin
\marginparwidth 0.5in
\textwidth 6.5in

\parindent 0in
\parskip 1.5ex

\newcommand{\homework}[2]{
	\noindent
	\begin{center}
		\framebox{
			\vbox{
				\hbox to 6.50in { {\bf NYU CS-GY 6763: Algorithmic Machine Learning 
						and Data Science} \hfill Spring 2025 }
				\vspace{4mm}
				\hbox to 6.50in { {\Large \hfill Homework #2  \hfill} }
				\vspace{2mm}
				\hbox to 6.50in { {Name: #2 \hfill} }
			}
		}
	\end{center}
	\vspace*{4mm}
}

\begin{document}
	
\section*{Problem 1 }
	1. \hspace{1em} Assume our hash table has $cm^2$ slots for  large constant $c$. Conditioned on the event that none of the first $i-1$ items collided when inserted in the table, the probability that the $i^\text{th}$ item does not causes a collision is
	\begin{align*}
		1 - \frac{i-1}{cm^2}.
	\end{align*}
In particular, when the $i^\text{th}$ item is inserted, the table has $i-1$ filled slots out of $cm^2$ total. 

From the expression above, we can bound the probability there are no collisions by:
	\begin{align*}
	\left(1 - \frac{0}{cm^2}\right)\cdot \left(1 - \frac{1}{cm^2}\right)\cdot \left(1 - \frac{2}{cm^2}\right) \cdot \ldots \cdot \left(1 - \frac{m-1}{cm^2}\right) < \left(1 - \frac{1}{cm}\right)^m.
	\end{align*}
	The inequality follow from the fact the $i/m < 1$ for all $i \leq m-1$. 

	As hinted, for $cm \geq 2$, $(1 - \frac{1}{cm})^{cm} \geq 1/2e$. So, $(1 - \frac{1}{cm})^m = \left((1 - \frac{1}{cm})^{cm}\right)^{1/c} \geq (1/2e)^{1/c}$.  Choosing $c \geq 17$ yields $(1/2e)^{1/c} \geq 9/10$. So, as long as the table has $\geq 17m^2$ slots, there is no collision with probability $> 9/10$. 
	
	\medskip\noindent 2. \hspace{1em}
	By linearity of expectation, we have the $\E[\|x\|_2^2]  = \sum_{i=1}^d \E[x_i^2]$. So,  we focus on bounding $\E[x_i^2]$ for a single $i$. We have that $x_i = \sum_{j=1}^n s_j$ where:
	\begin{align*}
		s_j = \begin{cases}
			0 & \text{with probability } 1 - 1/d \\
			-1 & \text{with probability } 1/2d \\
			+1 & \text{with probability } 1/2d 
		\end{cases}
	\end{align*}
	So we calculate the expectation:
	\begin{align*}
		\E[x_i^2] = \E\left[\left(\sum_{j=1}^n s_j\right)^2\right] &= \E\left[s_1^2 + \ldots + s_n^2 + s_1s_2 + \ldots + s_js_k + \ldots + s_{n-1} s_n\right] \\
		&= \E[s_1^2] + \ldots + \E[s_n^2] + \E[s_1s_2] + \ldots + \E[s_js_k] + \ldots + \E[s_{n-1}s_n]
	\end{align*}
We have that for each $s_j$, $\E[s_j] = 0$ and $s_j$ and $s_k$ are independent, so all of the terms of the form $\E[s_js_k]$ equal $0$. So we have that:
\begin{align*}
	\E[x_i^2] = \E[s_1^2] + \ldots + \E[s_n^2].
\end{align*}
It can be directly calculated that for each $j$, $\E[s_j^2] =  1\cdot \frac{1}{2d} + 1\cdot \frac{1}{2d} = \frac{1}{d}$. So $\E[x_i^2] = \frac{n}{d}$ and finally,
\begin{align*}
	\E[\|x\|_2^2]  = \sum_{i=1}^d \E[x_i^2] = d\cdot \frac{n}{d} = n.
\end{align*}
	

\section*{Problem 2}
There are many ways to solve this problem, including constructions that use multiple tables (like we did in class). I will provide a simple solution using just one table, $A$, with size $cm$, where $c$ is a large constant to be chosen later. Let $\tilde{f} = A(h(v))$ be the frequency estimate for item $v$ obtained from just that table.

To get the improved bound, we split up  $A(h(v))$ in a slightly refined way than what was done in class. The intuition is that, for large enough $c$, item $v$ will not collide with \emph{any} of the top $m$ most frequent items with high probability. Call these items $v_1, \ldots, v_m$. Call the less frequent items $v_{m+1}, v_{m+2}, \ldots$. We have:
\begin{align*}
	A(h(v)) = f(v) + \sum_{i=1,\ldots, m, v_i \neq v}\mathbbm{1}[h(v) = h(v_i)]\cdot f(v_i) + \sum_{i> m, v_i \neq v}\mathbbm{1}[h(v) = h(v_i)] f(y).
\end{align*}
The second two terms are our error terms. To bound the first, note that the event $\mathbbm{1}[h(v) = h(v_i)]$ only happens with probability $1/cm$ if our table has $cm$ slots in it. Moreover, $\mathbbm{1}[h(v) = h(v_i)]$ is independent from $\mathbbm{1}[h(v) = h(v_j)]$ for $i \neq j$. So, none of the events $\mathbbm{1}[h(v) = h(v_i)]$ happen with probability:
\begin{align*}
	(1 - 1/cm)^m = \left((1 - 1/cm)^{cm}\right)^{1/c}  \geq (1/2e)^{1/c}. 
\end{align*}
Choosing $c = 50$, we have that $(1/2e)^{1/c} \geq .95$, so
\begin{align*}
	\Pr\left[\sum_{i=1,\ldots, m, v_i \neq v}\mathbbm{1}[h(v) = h(v_i)]\cdot f(v_i) = 0\right] \geq 19/20. 
\end{align*}
I.e., with very high probability, the first error term is $0$. 

We now turn to the second error term. As in class, we have that: 
\begin{align*}
\E\left[\sum_{i> m, v_i \neq v}\mathbbm{1}[h(v) = h(v_i)] f(y)\right] = \frac{1}{cm} \sum_{i> m, v_i \neq v} f(y) \leq \frac{1}{cm}\cdot C = \frac{1}{50m}\cdot C.
\end{align*}
Accordingly, by Markov's inequality, 
\begin{align*}
\Pr\left[\sum_{i> m, v_i \neq v}\mathbbm{1}[h(v) = h(v_i)] f(y) \geq \frac{1}{m}C\right] \leq 1/50. 
\end{align*}
We conclude by applying a union bound: the probability that the first error term is non-zero \textit{or} the second error term is $\geq \frac{1}{m} C$ is upper bounded by $1/20 + 1/50 \leq 1/10$. Accordingly, we conclude
\begin{align*}
\Pr\left[\tilde{f}(v) -f(v) \geq \frac{1}{m}C\right] \leq 1/10.
\end{align*}

\section*{Problem 3}
	\smallskip\noindent 1.\hspace{1em}
	Split the people being tested into $C$ equal groups with $n/C$ people each. At most $k$ of the groups will test positive since, in the worst case, each infected person will be in a different group. For each positive group, we need to rerun $n/C$ individual tests. So the number of tests run is at most $C + \frac{n}{C}\cdot k$. If we set $C = \sqrt{nk}$ then the total number of tests is:
	\begin{align*}
		C + \frac{n}{C}\cdot k = \sqrt{nk} + \frac{n}{\sqrt{n}\sqrt{k}}k = 2\sqrt{nk},
	\end{align*}  as desired. 

\smallskip\noindent 2.\hspace{1em}
Assume $q = c\log n$ for sufficiently large constant $c$ and assume $C = 2k$. Consider any single individual who is negative. For this individual to be falsely reported positive, they would need to test positive in all $q$ group tests they participates in. 

What is the probability they test positive in any one group test? For this to happen, there would need to be a positive individual in that group. Let $Z$ be the number of positive individuals in any given group. Each group has size $\frac{n}{2k}$, so by linearity of expectation, $\E[Z] = \frac{n}{2k}\cdot \frac{k}{n} = \frac{1}{2}$, since the probability that any group member is positive is $k/n$. Then by Markov's inequality, $\Pr[Z \geq 1] \leq \frac{1}{2}$. I.e., with probability $\geq \frac{1}{2}$ any given group contains \emph{no infected individuals}. 

So, the probability our negative individual has a positive person in all $q = c\log n$ of their groups is less than $\frac{1}{2}^{q} = \frac{1}{n^c}$, which is $\leq \frac{1}{10n}$ for sufficiently large $c$. Thus, the probability any given negative individual gets a false positive result is $\leq \frac{1}{10n}$. There are $n - k < n$ negative individuals and by a union bound, the probability \emph{any} of them gets a false positive test is $\leq \frac{1}{10n}\cdot n \leq \frac{1}{10}$. In other words, we get no false positives with probability $\geq 9/10$. 

\smallskip\noindent 3.\hspace{1em} \textbf{Informal}: The lower bound is via a counting argument. One way of framing our goal is that we need to return a bit string of length $n$ which has a $1$ in exactly $k$ places and zeros everywhere else, with the $1$'s indicating the people we believe are infected, and the $0$'s indicating uninfected. 

Before starting the testing scheme, we don't know what the correct bit string is, and there are ${n \choose k}$ different possibilities. When we run any testing scheme, we will receive a binary reponse to every test $t_1, \ldots, t_T$ and from that response will decide how to run the next test. In the end, we will decide on which bit string of the ${n \choose k}$ different possibilities is correct. The solution we return must be a function of $t_1, \ldots, t_T$: it could also depend on what subsets of people were tested, but at any time $i$, that must be a function of $t_1, \ldots, t_{i-1}$, via induction. 

So, to obtain a correct answer after $T$ tests, it must be that the number of possible test results $t_1, \ldots, t_T$ exceeds the number of possible solutions ${n \choose k}$. In other words that $2^t \geq {n \choose k}$. Taking logs on both sides, and noting that $\log{n \choose k}  = O(k\log(n/k))$ gives the result.


\section*{Problem 4}
\smallskip\noindent 1.\hspace{1em} Recall that $\tilde{n} = \frac{m(m-1)}{2D}$, where $m$ is the number of samples collected, and $D$ is the number of duplicates observed. To prove that $(1-\epsilon) n \leq \tilde{n} \leq (1+\epsilon) n$ we first claim that it suffices to show:
\begin{align}
	\label{to:prove}
	(1-\epsilon/2)\E[D] \leq D \leq (1+\epsilon/2)\E[D].
\end{align}
In particular, if this is the case, then:
\begin{align*}
	\frac{1}{1+\epsilon/2}\frac{1}{\E[D]} \leq \frac{1}{D} \leq \frac{1}{1-\epsilon/2}\frac{1}{\E[D]}, 
\end{align*}
Using the inequalities from the start of Lecture 2, it follows that for $0 < \epsilon \leq 1$,
\begin{align*}
	(1-\epsilon/2)\frac{1}{\E[D]} \leq \frac{1}{D} \leq (1+\epsilon)\frac{1}{\E[D]}.
\end{align*}
Multiplying all sides of the inequalities by $m(m-1)/2$, we conclude that $(1-\epsilon) n \leq \tilde{n} \leq (1+\epsilon) n$

So, we focus on proving \eqref{to:prove}. To do so, we will bound the variance of $D$ and use Chebyshev's inequality. Use $s_1, \ldots, s_m$ to denote our $m$ samples. We have that:
\begin{align*}
	D = \sum_{i < j} \mathbbm{1}[s_i = s_j].
\end{align*}
Note that the terms in the sum are pairwise independent. In particular, the event that $s_i = s_j$ does not effect the probability that $s_k = s_j$ for some other value of $k$. As discussed in class, the terms are not mutually independent, since the event that $s_i = s_j$ and $s_j = s_k$ implies that $s_i = s_k$. Regardless, pairwise independence is all we need to apply linearity of variance: 
\begin{align*}
	\var[D] = \sum_{i < j} \var[\mathbbm{1}[s_i = s_j]].
\end{align*}
$\mathbbm{1}[s_i = s_j]$ is a binary random variable equal to $1$ with probability $1/n$, so its variance is equal to $\frac{1}{n} - \frac{1}{n^2}$. We conclude that 
\begin{align*}
	\var[D] \leq \sum_{i < j} \frac{1}{n} = \binom{m}{2}\cdot \frac{1}{n} = \frac{m(m-1)}{2n}.
\end{align*}
Note that this is exactly equal to $\E[D]$. So, we have via Chebyshev's inequality that:
\begin{align*}
	\Pr\left[|D - \E[D] \leq \sqrt{10}\sqrt{\E[D]}\right] \leq \frac{1}{10}.
\end{align*}
To prove \eqref{to:prove}, we need to choose $m$ large enough so that $\sqrt{10}\sqrt{\E[D]} \leq \epsilon \E[D]$. I.e., we need
\begin{align*}
	\sqrt{\frac{m(m-1)}{2n}} &\geq \frac{\sqrt{10}}{\epsilon}\\
	m(m-1) &\geq \frac{20 n}{\epsilon^2} \\
	m \geq \sqrt{\frac{40n}{\epsilon^2}}.
\end{align*}
This proves the claim.

\smallskip\noindent 3.\hspace{1em} For the mark-and-recapture estimator to be accurate as the number of samples goes to infinity, the key property we required was that $\E[D] = \frac{m(m-1)}{2n}$. Since we use the estimator $\tilde{n} = \frac{m(m-1)}{2D}$, we will obtain an \emph{underestimate} in the limit if: 
\begin{align*}
\E[D] > \frac{m(m-1)}{2n}.
\end{align*}
We claim that this is the case for data items drawn from \emph{any} non-uniform distribution -- Wikipedia's particular distribution doesn't matter too much. 

In particular, suppose the sample procedure selects item $z$ with probability $p_z$. If we collect samples $s_1, \ldots, s_m$, the expected number of duplicates is: 
\begin{align*}
\E[D] = \E\left[\sum_{i<j} \mathbbm{1}[s_i = s_j]\right] = \frac{m(m-1)}{2}\cdot \E\left[\mathbbm{1}[s_1 = s_2]\right].
\end{align*}
Here I am using that $\mathbbm{1}[s_i = s_j]$ is the idenitically distributed for all $i,j$, so I can just fixed $i$ and $j$ to be $1,2$ arbitrarily. 

What is $\E\left[\mathbbm{1}[s_1 = s_2]\right]$? We can write this as follows:
\begin{align*}
	\E\left[\mathbbm{1}[s_1 = s_2]\right] = \Pr[s_1 = s_2] = \sum_{z=1}^n \Pr[s_1 = z \text{ and } s_2 = z] = \sum_{z=1}^n p_z^2. 
\end{align*}
So, we conclude that: 
\begin{align*}
	\E[D] =  \frac{m(m-1)}{2}\cdot \sum_{z=1}^n p_z^2.
\end{align*}
We will obtain an underestimate if $\sum_{z=1}^n p_z^2 > n$. Note that it is exactly equal to $n$ in the uniform case when $p_1, \ldots, p_n = \frac{1}{n}$. There are a number of ways to see that it is larger for any other choice of $n$. 

One way is via an ``exchange argument". Suppose our probabilities are not uniform, so there are at least two items $a$ and $b$ such that $p_a \neq p_b$. Now, consider modifying the distribution so that $p_a$ and $p_b$ are both replaced with their average $p_a' = p_b' = \frac{p_a+p_b}{2}$. Then, after this change, the summation $\sum_{z=1}^n p_z^2$ changes by:
\begin{align*}
	{p_a'}^2 + {p_b'}^2 - p_a^2 - p_b^2 = 2\cdot \frac{p_a^2 + 2p_ap_b + p_b^2}{4} - p_a^2 - p_b^2 = p_ap_b - p_a^2/2 -p_b^2/2 = -(p_a/\sqrt{2} - p_b/\sqrt{2})^2.
\end{align*}
This change is always \emph{negative}, meaning that the sum decreases. So, take any non-uniform distribution and make it more uniform by averaging two of the non-equal probabilities and you will \emph{strictly decrease} $\sum_{z=1}^n p_z^2$. We conclude that the sum is always strictly larger than the value of $n$ obtained by the uniform distribution. 

\smallskip\noindent 4.\hspace{1em} 
In Wikipedia's case, the probabilities $p_1, \ldots, p_n$ are themselves random. So, to understand, $\sum_{z=1}^n p_z^2$, it makes sense to consider:
\begin{align*}
\E\left[\sum_{z=1}^n p_z^2\right] = n \cdot \E[p_z^2]. 
\end{align*} 
For simplicity, imagine that we are assigning Wikipedia pages random numbers between $[0,1]$
We can compute the expecation of $p_z$ using a similar approach as in Lecture 1 by first trying to understand $\Pr[p_z \geq t]$ for a fixed threshold $t$. In particular, for $p_z$ to be $\geq t$, it must be that no other page lands within distance $t$ of $r_z$. Imageine $r_z$ is chosen first. Then the chance this happens is $(1-t)^{n-1}$. So have have:
\begin{align*}
\Pr[p_z \geq t] &= (1-t)^{n-1}
\end{align*} 
or equivalently 
\begin{align*}
	\Pr[p_z^2 \geq t] &= (1-\sqrt{t})^{n-1}
\end{align*} 
We thus have:
\begin{align*}
	\E[p_z^2] = \int_{0}^1 \Pr[p_z^2 \geq t] dt = \int_{0}^1 (1-\sqrt{t})^{n-1} dt = \frac{2}{n^2 + n}. 
\end{align*} 
It follows that:
\begin{align*}
	\E\left[\sum_{z=1}^n p_z^2\right] = n \cdot \E[p_z^2] = \frac{2}{n+1}.
\end{align*} 
For large $n$, this is very close to $2/n$. So, $\E[D]$ will be almost exactly twice as large as when the probabilities are exactly uniform (in which case $\sum_{z=1}^n p_z^2 = 1/n$).

\end{document}