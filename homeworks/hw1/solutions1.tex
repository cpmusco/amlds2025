\documentclass[11pt]{article}
\usepackage{fullpage}
\usepackage{amsmath,amsfonts,amsthm,amssymb}
\usepackage{url}
\usepackage[demo]{graphicx}
\usepackage{caption} 
\usepackage{algpseudocode}
\usepackage{bbm}
\usepackage{float}
\usepackage{framed}
\usepackage{enumerate}
\usepackage{color}
\usepackage[colorlinks=true, linkcolor=red, urlcolor=blue, citecolor=blue]{hyperref}

\DeclareMathOperator*{\E}{\mathbb{E}}
\DeclareMathOperator*{\R}{\mathbb{R}}
\DeclareMathOperator{\var}{Var}

\topmargin 0pt
\advance \topmargin by -\headheight
\advance \topmargin by -\headsep
\textheight 8.9in
\oddsidemargin 0pt
\evensidemargin \oddsidemargin
\marginparwidth 0.5in
\textwidth 6.5in

\parindent 0in
\parskip 1.5ex

\newcommand{\homework}[2]{
	\noindent
	\begin{center}
		\framebox{
			\vbox{
				\hbox to 6.50in { {\bf NYU CS-GY 6763: Algorithmic Machine Learning 
						and Data Science} \hfill Fall 2023 }
				\vspace{4mm}
				\hbox to 6.50in { {\Large \hfill Homework #1  \hfill} }
				\vspace{2mm}
				\hbox to 6.50in { {Name: #2 \hfill} }
			}
		}
	\end{center}
	\vspace*{4mm}
}

\begin{document}
	
	\homework{1}{Solution key}
	
	\section*{Problem 1 }
	(a) \hspace{1em} Assume our hash table has $cm^2$ slots for  large constant $c$. The probability that the $i^\text{th}$ item does not causes a collision when inserted is greater than or equal to:
	\begin{align*}
		1 - \frac{i-1}{cm^2}.
	\end{align*}
To see that this is that case, note that the table only has \emph{at most} $i-1$ filled slots out of $cm^2$ total, so the probablilty of a collision is $\leq \frac{i-1}{cm^2}$. 
Using the expression above, we have that the probability there are no collisions at all is greater than:
	\begin{align*}
	\left(1 - \frac{0}{cm^2}\right)\cdot \left(1 - \frac{1}{cm^2}\right)\cdot \left(1 - \frac{2}{cm^2}\right) \cdot \ldots \cdot \left(1 - \frac{m-1}{cm^2}\right) \geq \left(1 - \frac{1}{cm}\right)^m.
	\end{align*}
	As hinted, for $cm \geq 2$, $(1 - \frac{1}{cm})^{cm} \geq 1/2e$. So, $(1 - \frac{1}{cm})^m = \left((1 - \frac{1}{cm})^{cm}\right)^{1/c} \geq (1/2e)^{1/c}$.  Choosing $c \geq 17$ yields $(1/2e)^{1/c} \geq 9/10$. So, as long as the table has $\geq 17m^2$ slots, there is no collision with probability $\geq 9/10$. 
	
	\smallskip\noindent (b)
	By linearity of expectation, we have the $\E[\|x\|_2^2]  = \sum_{i=1}^d \E[x_i^2]$, so we focus on bounding $\E[x_i^2]$ for a single $i$. We have that $x_i = \sum_{j=1}^n s_j$ where:
	\begin{align*}
		s_j = \begin{cases}
			0 & \text{with probability } 1 - 1/d \\
			-1 & \text{with probability } 1/2d \\
			+1 & \text{with probability } 1/2d 
		\end{cases}
	\end{align*}
	So we calculate the expectation:
	\begin{align*}
		\E[x_i^2] = \E\left[\left(\sum_{j=1}^n s_j\right)^2\right] &= \E\left[s_1^2 + \ldots + s_n^2 + s_1s_2 + \ldots + s_js_k + \ldots + s_{n-1} s_n\right] \\
		&= \E[s_1^2] + \ldots + \E[s_n^2] + \E[s_1s_2] + \ldots + \E[s_js_k] + \ldots + \E[s_{n-1}s_n]
	\end{align*}
We have that for each $s_j$, $\E[s_j] = 0$ and $s_j$ and $s_k$ are independent, so all of the terms of the form $\E[s_js_k]$ equal $0$. So we have that:
\begin{align*}
	\E[x_i^2] = \E[s_1^2] + \ldots + \E[s_n^2].
\end{align*}
It can be directly calculated that for each $j$, $\E[s_j^2] =  1\cdot \frac{1}{2d} + 1\cdot \frac{1}{2d} = \frac{1}{d}$. So $\E[x_i^2] = \frac{n}{d}$ and finally,
\begin{align*}
	\E[\|x\|_2^2]  = \sum_{i=1}^d \E[x_i^2] = d\cdot \frac{n}{d} = n.
\end{align*}
	
	
\section*{Problem 2}
	We use the notation from the Lecture 1 notes. Consider a single table $A$ and let $\tilde{f} = A(h(v))$ be the frequency estimate for just that table. It suffices to show that, with probability $\geq c$ for any constant $c$, 
	\begin{align*}
		f(v) \leq \tilde{f} \leq f(v) + \frac{1}{m} C.
	\end{align*}
As long as this is the case, then we can improve the success probability to $9/10$ by increasing the number of tables used to $\log_{1-c}(1/10) = O(1)$, exactly as was done in class for the standard Count-Min analysis. 

To get the improved bound, we split up  $A(h(v))$ in a slightly refined way. The intuition is that, with high probability, item $v$ will not collide with \emph{any} of the top $m$ most frequent items. Call these items $v_1, \ldots, v_m$. Call the less frequent items $v_{m+1}, v_{m+2}, \ldots$. We have:
\begin{align*}
	A(h(v)) = f(v) + \sum_{i=1,\ldots, m, v_i \neq v}\mathbbm{1}[h(v) = h(v_i)]\cdot f(v_i) + \sum_{i> m, v_i \neq v}\mathbbm{1}[h(v) = h(v_i)] f(y).
\end{align*}
The second two terms are our error terms. To bound the first, note that the event $\mathbbm{1}[h(v) = h(v_i)]$ only happens with probability $1/m$ if our table has $m$ slots in it. Moreover, $\mathbbm{1}[h(v) = h(v_i)]$ is independent from $\mathbbm{1}[h(v) = h(v_j)]$ for $i \neq j$. So, none of the events $\mathbbm{1}[h(v) = h(v_i)]$ happen with probability:
\begin{align*}
	(1 - 1/m)^m \geq 1/2e \approx .18.
\end{align*}
Moreover, using the same Markov's inequality analysis from class, we have that the final term $\sum_{y\notin v\cup \{v_1, \ldots, v_m\}}\mathbbm{1}[h(v) = h(v_i)] f(y)$ is upper bounded by $\frac{10}{m} \sum_{y\notin v\cup \{v_1, \ldots, v_m\}}f(y) = \frac{10}{m} \cdot C$ with probability $9/10$. So, by a union bound, with probability at least $.08$, we have that:
\begin{align*}
	\sum_{i=1,\ldots, m, v_i \neq v}\mathbbm{1}[h(v) = h(v_i)]\cdot f(v_i) &= 0
\end{align*}
and 
\begin{align*}
	\sum_{y\notin v\cup \{v_1, \ldots, v_m\}}\mathbbm{1}[h(v) = h(v_i)] f(y) \leq \frac{10}{m} \cdot C.
\end{align*}
Adjusing constants on $m$ gives the desired result. 

	
	\section*{Problem 3}
	\smallskip\noindent 1.\hspace{1em}
	Split the people being tested into $C$ equal groups with $n/C$ people each. At most $k$ of the groups will test positive since, in the worst case, each infected person will be in a different group. For each positive group, we need to rerun $n/C$ individual tests. So the number of tests run is at most $C + \frac{n}{C}\cdot k$. If we set $C = \sqrt{nk}$ then the total number of tests is:
	\begin{align*}
		C + \frac{n}{C}\cdot k = \sqrt{nk} + \frac{n}{\sqrt{n}\sqrt{k}}k = 2\sqrt{nk},
	\end{align*}  as desired. 

\smallskip\noindent 2.\hspace{1em}
Assume $q = c\log n$ for sufficiently large constant $c$ and assume $C = 2k$. Consider any single individual who is negative for COVID. For this individual to be falsely reported positive, they would need to test positive in all $q$ group tests they participates in. 

What is the probability they test positive in any one group test? For this to happen, there would need to be a positive individual in that group. Let $Z$ be the number of positive individuals in any given group. Each group has size $\frac{n}{2k}$, so by linearity of expectation, $\E[Z] = \frac{n}{2k}\cdot \frac{k}{n} = \frac{1}{2}$, since the probability that any group member is positive is $k/n$. Then by Markov's inequality, $\Pr[Z \geq 1] \leq \frac{1}{2}$. I.e., with probability $\geq \frac{1}{2}$ any given group contains \emph{no infected individuals}. 

So, the probability our negative individual has a positive person in all $q = c\log n$ of their groups is less than $\frac{1}{2}^{q} = \frac{1}{n^c}$, which is $\leq \frac{1}{10n}$ for sufficiently large $c$. Thus, the probability any given negative individual gets a false positive result is $\leq \frac{1}{10n}$. There are $n - k < n$ negative individuals and by a union bound, the probability \emph{any} of them gets a false positive test is $\leq \frac{1}{10n}\cdot n \leq \frac{1}{10}$. In other words, we get no false positives with probability $\geq 9/10$. 

\smallskip\noindent 3.\hspace{1em} \textbf{Informal}: The lower bound is via a counting argument. One way of framing our goal is that we need to return a bit string of length $n$ which has a $1$ in exactly $k$ places and zeros everywhere else, with the $1$'s indicating the people we believe are infected, and the $0$'s indicating uninfected. 

Before starting the testing scheme, we don't know what the correct bit string is, and there are ${n \choose k}$ different possibilities. When we run any testing scheme, we will receive a binary reponse to every test $t_1, \ldots, t_T$ and from that response will decide how to run the next test. In the end, we will decide on which bit string of the ${n \choose k}$ different possibilities is correct. The solution we return must be a function of $t_1, \ldots, t_T$: it could also depend on what subsets of people were tested, but at any time $i$, that must be a function of $t_1, \ldots, t_{i-1}$, via induction. 

So, to obtain a correct answer after $T$ tests, it must be that the number of possible test results $t_1, \ldots, t_T$ exceeds the number of possible solutions ${n \choose k}$. In other words that $2^t \geq {n \choose k}$. Taking logs on both sides, and noting that $\log{n \choose k}  = O(k\log(n/k))$ gives the result.

\section*{Problem 4}
The probability a coin flipped 100 times comes up $\geq 60$ times heads is the sum of the probabilities it comes up exactly 60 times heads, 61 times heads, ... 100 times heads. The probability a coin comes up $x$ time heads equals ${100 \choose x}\cdot (\frac{1}{2})^{100}$. So we have:
\begin{align*}
	\Pr[\geq 60 \text{ heads out of } 100] = \sum_{i=60}^{100} {100 \choose i}\cdot \left(\frac{1}{2}\right)^{100} = \mathbf{.028} \\
	\Pr[\geq 600 \text{ heads out of } 1000] = \sum_{i=600}^{1000} {1000 \choose i}\cdot \left(\frac{1}{2}\right)^{1000} = \mathbf{1.36 \cdot 10^{-10}} \\
\end{align*}
They need to take some care when computing these numbers! For most languages the $(\frac{1}{2})^{1000}$ won't be a problem: $1/2$ will be treated a floating point double, which can represent numbers as small as $10^-308$. $(\frac{1}{2})^{1000}$ is roughly $10^{-302}$ so all good :). On the other hand, there is a chance that the language tries to represent ${100 \choose i}$ as an integer, which will overflow... I check both Matlab and Python, which will work for different reasons. Matlab will represent the result as a floating point number. Python will detect the overflow and use an integer type with more bits to prevent it.

To compare to the Chernoff bounds, note that $\mu = 50$ and $\mu = 500$ for the two problems, and for both $\epsilon = .2$. Using the different bounds we get:
\begin{align*}
	\Pr[\geq 60 \text{ heads out of } 100]\leq .4029\\
	\Pr[\geq 60 \text{ heads out of } 100]\leq .3909
\end{align*}
and 
\begin{align*}
	\Pr[\geq 600 \text{ heads out of } 1000]\leq 1.12 \cdot 10^{-4}\\ 
	\Pr[\geq 600 \text{ heads out of } 1000]\leq  8.33\cdot 10^{-5}
\end{align*}
The take away is that, even though they improve on e.g. Chebyshev's inequality, Chernoff bounds are actually quite loose -- off by orders of magnitude in comparison the precise answer. There isn't a huge difference between the looser bound we typically use and the tighter bound you can find online though, which is good news. 

\section*{Problem 5}
1. Recall that $\tilde{n} = \frac{m(m-1)}{2D}$, where $m$ is the number of samples collected, and $D$ is the number of duplicates observed. Also recall that $n = \frac{m(m-1)}{2\E[D]}$. So, using an argument identical to the distinct elements analysis in Lecture 2, it suffices to show that:
\begin{align}
	\label{to:prove}
	(1-\epsilon/2)\E[D] \leq D \leq (1+\epsilon/2)\E[D].
\end{align}
In particular, if this is the case,
\begin{align*}
	\frac{1}{1+\epsilon/2}\frac{1}{\E[D]} \leq \frac{1}{D} \leq \frac{1}{1-\epsilon/2}\frac{1}{\E[D]}, 
\end{align*}
which implies that, for any $\epsilon \leq 1$,
\begin{align*}
	(1-\epsilon/2)\frac{1}{\E[D]} \leq \frac{1}{D} \leq (1+\epsilon)\frac{1}{\E[D]},
\end{align*}
and thus $(1-\epsilon) n \leq \tilde{n} \leq (1+\epsilon) n$.

So, we focus on proving \eqref{to:prove}. To do so, we will bound the variance of $D$ and use Chebyshev's inequality. Use $s_1, \ldots, s_m$ to denote our $m$ samples. We have that:
\begin{align*}
	D = \sum_{i < j} \mathbbm{1}[s_i = s_j].
\end{align*}
Note that the terms in the sum are pairwise independent. In particular, the event that $s_i = s_j$ does not effect the probability that $s_k = s_j$ for some other value of $k$. Interesting the terms are not three-wise independent, since the event that $s_i = s_j$ and $s_j = s_k$ implies that $s_i = s_k$. Regardless, pairwise independence is all we need to apply linearity of variance: 
\begin{align*}
	\var[D] = \sum_{i < j} \var[\mathbbm{1}[s_i = s_j]].
\end{align*}
$\mathbbm{1}[s_i = s_j]$ is a binary random variable equal to $1$ with probability $1/n$, so its variance is equal to $\frac{1}{n} - \frac{1}{n^2}$. We conclude that 
\begin{align*}
	\var[D] \leq \sum_{i < j} \frac{1}{n} = \binom{m}{2}\cdot \frac{1}{n} = \frac{m(m-1)}{2n}.
\end{align*}
Note that this is exactly equal to $\E[D]$. So, we have via Chebyshev's inequality that:
\begin{align*}
	\Pr\left[|D - \E[D] \leq \sqrt{10}\sqrt{\E[D]}\right] \leq \frac{1}{10}.
\end{align*}
To prove \eqref{to:prove}, we need to choose $m$ large enough so that $\sqrt{10}\sqrt{\E[D]} \leq \epsilon \E[D]$. I.e., we need
\begin{align*}
	\sqrt{\frac{m(m-1)}{2n}} &\geq \frac{\sqrt{10}}{\epsilon}\\
	m(m-1) &\geq \frac{20 n}{\epsilon^2} \\
	m \geq \sqrt{\frac{40n}{\epsilon^2}}.
\end{align*}
This proves the claim.

% 2. My code for solving the problem is here: \url{https://colab.research.google.com/drive/1htTKFbA_LFVsV0RNFzf4X_RXk1coe6xK?usp=sharing}. We don't expect their estimator to work perfectly (see bonus on next problem set). Try to grade this one generously -- as long as they have what looks like a correct estimator and get an estimate that's on the same order of magnitude as 6.7 million, it's okay if the explanation is not perfect. 
\end{document}