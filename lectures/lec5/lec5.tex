%!LW recipe=latexmk-xelatex
\documentclass[compress]{beamer}

\usetheme[block=fill]{metropolis}

\usepackage{graphicx} % Allows including images
\usepackage{amsmath,amsfonts,amsthm,amssymb}
\usepackage{color}
\usepackage{xcolor,cancel}
\definecolor{mDarkBrown}{HTML}{604c38}
\definecolor{mDarkTeal}{HTML}{23373b}
\definecolor{mLightBrown}{HTML}{EB811B}
\definecolor{mMediumBrown}{HTML}{C87A2F}
\definecolor{mygreen}{HTML}{98C2B9}
\definecolor{myyellow}{HTML}{DFD79C}
\definecolor{myblue}{HTML}{8CA7CC}
\definecolor{kern}{HTML}{8CC2B7}


\usepackage{float}
\usepackage{framed}
\usepackage{epsfig}
\usepackage{graphicx}
\usepackage{subcaption}
\usepackage{ulem}
\usepackage{hhline}
\usepackage{multirow}
\usepackage{comment}   
\usepackage{bbm}
\usepackage{tikz}   
\def\Put(#1,#2)#3{\leavevmode\makebox(0,0){\put(#1,#2){#3}}}
\newcommand*\mystrut[1]{\vrule width0pt height0pt depth#1\relax}
\newcommand{\eqdef}{\mathbin{\stackrel{\rm def}{=}}}


\newcommand{\bs}[1]{\boldsymbol{#1}}
\newcommand{\bv}[1]{\mathbf{#1}}
\newcommand{\R}{\mathbb{R}}
\newcommand{\E}{\mathbb{E}}

\DeclareMathOperator*{\argmin}{arg\,min}
\DeclareMathOperator*{\argmax}{arg\,max}
\DeclareMathOperator{\nnz}{nnz}
\DeclareMathOperator{\Var}{Var}
\DeclareMathOperator{\sinc}{sinc}
\DeclareMathOperator{\sign}{sign}
\DeclareMathOperator{\dist}{dist}
\DeclareMathOperator{\mv}{mv}
\DeclareMathOperator{\sgn}{sgn}
\DeclareMathOperator{\step}{step}
\DeclareMathOperator{\gap}{gap}
\DeclareMathOperator{\poly}{poly}
\DeclareMathOperator{\tr}{tr}
\DeclareMathOperator{\orth}{orth}
\newcommand{\norm}[1]{\|#1\|}
\captionsetup[subfigure]{labelformat=empty}
\captionsetup[figure]{labelformat=empty}
\DeclareMathOperator*{\lmin}{\lambda_{min}}
\DeclareMathOperator*{\lmax}{\lambda_{max}}

\newcommand{\specialcell}[2][c]{%
  \begin{tabular}[#1]{@{}c@{}}#2\end{tabular}}
\newcommand{\specialcellleft}[2][c]{%
\begin{tabular}[#1]{@{}l@{}}#2\end{tabular}
}

\usepackage{tabstackengine}
\stackMath


%----------------------------------------------------------------------------------------
%	TITLE PAGE
%----------------------------------------------------------------------------------------

\title{CS-GY 6763: Lecture 5 \\ Dimensionality reduction, near neighbor search in high dimensions}
\author{NYU Tandon School of Engineering, Prof. Christopher Musco}
\date{}

\begin{document}

\begin{frame}
	\titlepage 
\end{frame}

\metroset{titleformat=smallcaps}


\begin{frame}
	\frametitle{dimensionality reduction}
	Despite all our warning from last class that low-dimensional space looks nothing like high-dimensional space, next we are going to learn about how to \textbf{compress high dimensional vectors to low dimensions.} 
	
	We will be very careful not to compress things \emph{too} far. An extremely simple method known as Johnson-Lindenstrauss Random Projection pushes right up to the edge of how much compression is possible. 
\end{frame}


\begin{frame}
	\frametitle{euclidean dimensionality reduction}
	\begin{lemma}[Johnson-Lindenstrauss, 1984]
		For any set of $n$ data points $\bv{q}_1,\ldots, \bv{q}_n \in \R^d$ there exists a \emph{linear map} $\Pi: \R^d \rightarrow \R^k$ where $k = O\left(\frac{\log n}{\epsilon^2}\right)$ such that \emph{for all $i,j$},
		\begin{align*}
			(1-\epsilon)\|\bv{q}_i - \bv{q}_j\|_2 \leq \|\bs{\Pi}\bv{q}_i - \bs{\Pi}\bv{q}_j\|_2 \leq (1+\epsilon)\|\bv{q}_i - \bv{q}_j\|_2.
		\end{align*}
	\end{lemma}
	\begin{center}
		\includegraphics[height=.45\textheight]{jl_sketch.png}
	\end{center}
\end{frame}

\begin{frame}
	\frametitle{euclidean dimensionality reduction}
	This is equivalent to: 
	\begin{lemma}[Johnson-Lindenstrauss, 1984]
		For any set of $n$ data points $\bv{q}_1,\ldots, \bv{q}_n \in \R^d$ there exists a \emph{linear map} $\Pi: \R^d \rightarrow \R^k$ where $k = O\left(\frac{\log n}{\epsilon^2}\right)$ such that \emph{for all $i,j$},
		\begin{align*}
			(1-\epsilon)\|\bv{q}_i - \bv{q}_j\|_2^{\mathbf{\alert{2}}} \leq \|\bs{\Pi}\bv{q}_i - \bs{\Pi}\bv{q}_j\|_2^{\mathbf{\alert{2}}} \leq (1+\epsilon)\|\bv{q}_i - \bv{q}_j\|_2^{\mathbf{\alert{2}}}.
		\end{align*}
	\end{lemma}
	because for small $\epsilon$, $(1+\epsilon)^2 = 1 + O(\epsilon)$ and $(1-\epsilon)^2 = 1 - O(\epsilon)$.
\end{frame}

\begin{frame}
	\frametitle{tons of applications}
	\textbf{Make pretty much any computation involving vectors faster and more space efficient.}
	\begin{itemize}
		\item Faster vector search (used in image search, AI-based web search, Retrieval Augmented Generation (RAG), etc.).
		\item Faster machine learning (today we will see an application to speeding up clustering).
		\item Faster numerical linear algebra. 
	\end{itemize}
	\begin{center}
		\alert{\textbf{Only useful if we can explicity construct a JL map $\bs{\Pi}$ and apply efficiently to vectors.}}
	\end{center}
\end{frame}

\begin{frame}
	\frametitle{euclidean dimensionality reduction}
	\begin{center}
		Remarkably, $\Pi$ can be chosen \emph{completely at random}!
	\end{center}
	\textbf{One possible construction:} Random Gaussian.
	\begin{align*}
		\bs{\Pi}_{i,j} = \frac{1}{\sqrt{k}} \mathcal{N}(0,1)
	\end{align*}
	The map $\bs{\Pi}$ is \textbf{\alert{oblivious to the data set}}. This stands in contrast to other vector compression methods you might know like PCA.
	
	[Indyk, Motwani 1998] [Arriage, Vempala 1999] [Achlioptas 2001] [Dasgupta, Gupta 2003].
	
	Many other possible choices suffice -- you can use random $\{+1,-1\}$ variables, sparse random matrices, pseudorandom $\Pi$. Each with different advantages. 
\end{frame}

\begin{frame}
	\frametitle{randomized jl constructions}
	\begin{center}
		Let $\bs{\Pi} \in \R^{k\times d}$ be chosen so that each entry equals $\frac{1}{\sqrt{k}}  \mathcal{N}(0,1)$.
		
		... or each entry equals $\frac{1}{\sqrt{k}}  \pm 1$ with equal probability.
	\end{center}
	\vspace{1em}
	
	\begin{columns}
		\begin{column}{0.5\textwidth}
			\includegraphics[width=\textwidth]{rand_gauss.png}
		\end{column}
		\begin{column}{0.5\textwidth}
			\includegraphics[width=\textwidth]{rand_sign.png}
		\end{column}
	\end{columns}
	
	\begin{center}
		A random orthogonal matrix $\bv{Q}$ also works. I.e. with $\bv{Q}\bv{Q}^T = \bv{I}_{k\times k}$. For this reason, the JL operation is often called a ``random projection", even though it technically is not a projection when $\bs{\Pi}'s$ entries are i.i.d.
	\end{center}
\end{frame}

\begin{frame}[t]
	\frametitle{random proection}
	Can anyone see why $\bs{\Pi}$ is similar to a projection matrix? I.e., a matrix satisfying  $\bv{Q}\bv{Q}^T = \bv{I}_{k\times k}$.
\end{frame}

\begin{frame}
	\frametitle{random projection}
	\begin{center}
		\includegraphics[width=.6\textwidth]{random_projection.png}
	\end{center}
	\textbf{Intuition:} Multiplying by a random matrix mimics the process of projecting onto a random $k$ dimensional subspace in $d$ dimensions.
\end{frame}


\begin{frame}
	\frametitle{application: the new paradigm for search}
	\begin{center}
				\includegraphics[width=.9\textwidth]{multimodal_embeddings.png}
	\end{center}
	Use neural network (BERT, CLIP, etc.) to convert documents, images, etc. to high dimensional vectors. Results matching search should have similar vector embeddings.
\end{frame}

\begin{frame}
	\frametitle{application: the new paradigm for search}
	\begin{center}
		\vspace{-.5em}
				\includegraphics[width=.8\textwidth]{vector_database.png}
				\vspace{-.5em}
	\end{center}
	Finding results for a query reduces to finding the nearest vector in a \emph{vector database}, with similarity typically measured by Euclidean distance. \textbf{This is a massive algorithmic challenge!}
\end{frame}

\begin{frame}
	\frametitle{another example of vector search}
	\begin{center}
			\textbf{Shazam} can match a song clip against a library of 8 million songs (32 TB of data) in a fraction of a second. Whole system based on vector embeddings + search.
			
			\only<1>{\includegraphics[width=.4\textwidth]{shazam.png}}
		\end{center}
	\vspace{-1em}
	\uncover<2->{
			\begin{figure}[h]
					\centering
					\begin{subfigure}[t]{0.45\textwidth}
							\centering
							\includegraphics[width=\textwidth]{spectrogram.png}
							\caption{Spectrogram extracted from audio clip.}
						\end{subfigure}
					~
					\begin{subfigure}[t]{0.45\textwidth}
							\centering
							\includegraphics[width=\textwidth]{spectrogramThresh.png}
							\caption{Processed spectrogram: used to construct audio ``fingerprint'' $\textbf{x}\in \R^d$.}
						\end{subfigure}
				\end{figure}
		}
\end{frame}

\begin{frame}
	\frametitle{vector search}
	Tons of new startups in the space (offering managed vector databases) and all major tech companies are franticly working on speeding up vector search. 
	\begin{center}
		\vspace{-.5em}
				\includegraphics[width=.6\textwidth]{startups.png}
				\vspace{-.5em}
	\end{center}
	\textbf{Two main ingredients:}
	\begin{enumerate}
	\item Vector indexing methods (second half of lecture).
	\item Vector compression methods (like Johnson-Lindenstrauss).
	\end{enumerate}
\end{frame}

\begin{frame}
	\frametitle{application: the new paradigm for search}
	Main computational cost is repeatedly computing $\|\bv{q}-\bv{x}_i\|_2$ for candidate result $\bv{x}_i$. 
	\begin{center}
		\vspace{-.5em}
				\includegraphics[width=.8\textwidth]{database_compression.png}
				\vspace{-.5em}
	\end{center}
	Vector compression leads to \emph{faster distance computations}. Not only is computational complexity reduced, but we can \emph{fit more database vectors in memory}.
\end{frame}

\begin{frame}
	\frametitle{euclidean dimensionality reduction}
	\begin{lemma}[Johnson-Lindenstrauss, 1984]
		Let $\bs{\Pi} \in \R^{k\times d}$ be chosen so that each entry equals $\frac{1}{\sqrt{k}}  \mathcal{N}(0,1)$, where $\mathcal{N}(0,1)$ denotes a standard Gaussian random variable. 
		
		If we choose $k = O\left(\frac{\log (n)}{\epsilon^2}\right)$, then with probability $99/100$, for \emph{for all $i,j$},
		\begin{align*}
			(1-\epsilon)\|\bv{q}_i - \bv{q}_j\|_2^{{{2}}} \leq \|\bs{\Pi}\bv{q}_i - \bs{\Pi}\bv{q}_j\|_2^{{{2}}} \leq (1+\epsilon)\|\bv{q}_i - \bv{q}_j\|_2^{{{2}}}.
		\end{align*}
	\end{lemma}
\end{frame}


\begin{frame}
	\frametitle{euclidean dimensionality reduction}
	\textbf{Intermediate result:}
	\begin{lemma}[Distributional JL Lemma]
		Let $\bs{\Pi} \in \R^{k\times d}$ be chosen so that each entry equals $\frac{1}{\sqrt{k}}  \mathcal{N}(0,1)$, where $\mathcal{N}(0,1)$ denotes a standard Gaussian random variable. 
		
		If we choose $k = O\left(\frac{\log(1/\delta)}{\epsilon^2}\right)$, then for \emph{any vector $\bv{x}$}, with probability $(1-\delta)$:
		\begin{align*}
			(1-\epsilon)\|\bv{x}\|_2^2 \leq \|\bs{\Pi}\bv{x}\|_2^2 \leq (1+\epsilon) \|\bv{x}\|_2^2
		\end{align*}
	\end{lemma}
	
	\begin{center}\alert{
			\textbf{Given this lemma, how do we prove the traditional Johnson-Lindenstrauss lemma?}}
	\end{center}
\end{frame}


\begin{frame}
	\frametitle{jl from distributional jl}
	We have a set of vectors $\bv{q}_1, \ldots, \bv{q}_n$. Fix $i,j \in 1,\ldots, n$. 
	
	Let $\bv{x} = \bv{q}_i - \bv{q}_j$. By linearity, $\bs{\Pi}\bv{x} = \bs{\Pi}(\bv{q}_i - \bv{q}_j) = \bs{\Pi}\bv{q}_i - \bs{\Pi}\bv{q}_j$.
	
	By the Distributional JL Lemma, with probability $1-\delta$,
	\begin{align*}
		(1-\epsilon)\|\bv{q}_i - \bv{q}_j\|_2 \leq \|\bs{\Pi}\bv{q}_i - \bs{\Pi}\bv{q}_j\|_2 \leq (1+\epsilon) \|\bv{q}_i - \bv{q}_j\|_2.
	\end{align*}
	Finally, set $\delta = \frac{1}{100n^2}$. Since there are $< n^2$ total $i,j$ pairs, by a union bound we have that with probability $99/100$, the above will hold \emph{for all} $i,j$, as long as we compress to:
	
	\begin{align*}
		k = O\left(\frac{\log(1/(1/100n^2))}{\epsilon^2}\right) = O\left(\frac{\log n}{\epsilon^2}\right) \text{ dimensions.}\qed
	\end{align*}
	
\end{frame}

\begin{frame}
	\frametitle{proof of distributional jl}
	Want to argue that, with probability $(1-\delta)$,
	\begin{align*}
		(1-\epsilon)\|\bv{x}\|_2^2 \leq |\bs{\Pi}\bv{x}\|_2^2 \leq (1+\epsilon)\|\bv{x}\|_2^2 
	\end{align*}
	
	\begin{center}
		\alert{\textbf{Claim}: $\E \|\bs{\Pi} \bv{x} \|_2^2 = \|\bv{x}\|_2^2.$}
	\end{center}
	
	\vspace{-1em}
	Some notation:
	\begin{center}
		\includegraphics[width=.6\textwidth]{jl_notation.png}
		
		So each $\bs{\pi}_i$ contains $\mathcal{N}(0,1)$ entries. 
	\end{center}
\end{frame}

\begin{frame}
	\frametitle{proof of distributional jl}
	\textbf{Intermediate Claim:} Let $\bs{\pi}$ be a length $d$ vector with $\mathcal{N}(0,1)$ entries. 
	\begin{align*}
		\E\left[\|\bs{\Pi} \bv{x} \|_2^2 \right]  = \E\left[\left(\langle\bs{\pi},\bv{x}\rangle\right)^2 \right] .
	\end{align*}
	%	\begin{align*}
		%		\|\bs{\Pi} \bv{x} \|_2^2 = \sum_i^k \bv{s}[i]^2 = \sum_i^k \left(\frac{1}{\sqrt{k}}\langle\bs{\pi}_i,\bv{x}\rangle\right)^2 = \frac{1}{k}\sum_i^k \left(\langle\bs{\pi}_i,\bv{x}\rangle\right)^2 
		%	\end{align*}
	%	\begin{align*}
		%		\E\left[\|\bs{\Pi} \bv{x} \|_2^2 \right] &= \frac{1}{k}\sum_i^k \E\left[\left(\langle\bs{\pi}_i,\bv{x}\rangle\right)^2 \right] \\
		%		& =\E\left[\left(\langle\bs{\pi}_i,\bv{x}\rangle\right)^2 \right] 
		%	\end{align*}
	\vspace{9em}
	\begin{block}{\vspace*{-3ex}}
		\small \textbf{Goal}: Prove $\E \|\bs{\Pi} \bv{x} \|_2^2 = \|\bv{x}\|_2^2$.
	\end{block}
\end{frame}

\begin{frame}
	\frametitle{proof of distributional jl}	
	\begin{align*}
		\langle\bs{\pi},\bv{x}\rangle = Z_1\cdot{x}[1] + Z_2\cdot{x}[2]  +  \ldots + Z_d\cdot{x}[d]
	\end{align*}
	where each $Z_1, \ldots, Z_d$ is a standard normal $\mathcal{N}(0,1)$. 
	
	We have that $Z_i \cdot{x}[i]$ is a normal $\mathcal{N}(0,{x}[i]^2)$ random variable.
	
	\vspace{5em}
	\begin{block}{\vspace*{-3ex}}
		\small \textbf{Goal}: Prove $\E \|\bs{\Pi} \bv{x} \|_2^2 = \|\bv{x}\|_2^2$. Established: $\E \|\bs{\Pi} \bv{x} \|_2^2 = \E\left[\left(\langle\bs{\pi},\bv{x}\rangle\right)^2 \right]$
	\end{block}
\end{frame}

\begin{frame}[t]
	\frametitle{stable random variables}
	\textbf{What type of random variable is $\langle\bs{\pi},\bv{x}\rangle$?}
	\begin{fact}[Stability of Gaussian random variables]
		\begin{align*}
			\mathcal{N}(\mu_1, \sigma_1^2) + \mathcal{N}(\mu_2, \sigma_2^2) =  \mathcal{N}(\mu_1 + \mu_2, \sigma_1^2 + \sigma_2^2)
		\end{align*}
	\end{fact}
	\begin{align*}
		\langle\bs{\pi},\bv{x}\rangle &= \mathcal{N}(0,{x}[1]^2) + \mathcal{N}(0,{x}[2]^2) + \ldots + \mathcal{N}(0,{x}[d]^2) \\ &= \mathcal{N}(0,\|\bv{x}\|_2^2). 
	\end{align*}
	
	So $\E \|\bs{\Pi} \bv{x} \|_2^2 = \E\left[\left(\langle\bs{\pi},\bv{x}\rangle\right)^2 \right] = \E\left[\mathcal{N}(0,\|\bv{x}\|_2^2)^2\right]  = \|\bv{x}\|_2^2$, as desired.
\end{frame}

\begin{frame}
	\frametitle{proof of distributional jl}
	Want to argue that, with probability $(1-\delta)$,
	\begin{align*}
		(1-\epsilon)\|\bv{x}\|_2^2 \leq \|\bs{\Pi}\bv{x}\|_2^2 \leq (1+\epsilon)\|\bv{x}\|_2^2 
	\end{align*}
	
	\begin{enumerate}
		\item $\E \|\bs{\Pi} \bv{x} \|_2^2 = \|\bv{x}\|_2^2$.
		\item Need to use a concentration bound.
	\end{enumerate}
	\begin{align*}
		\|\bs{\Pi} \bv{x} \|_2^2 = \frac{1}{k}\sum_{i=1}^k \left(\langle\bs{\pi}_i,\bv{x}\rangle\right)^2 = \frac{1}{k}\sum_{i=1}^k \mathcal{N}(0,\|\bv{x}\|_2^2)^2
	\end{align*}
	\alert{``Chi-squared random variable with $k$ degrees of freedom.''}
\end{frame}

\begin{frame}[t]
	\frametitle{concentration of chi-squared random variables}
	\begin{lemma} Let $H$ be a Chi-squared random variable with $k$ degrees of freedom. 
		\begin{align*}
			\Pr[|\E H - H| \geq \epsilon \E H] \leq 2 e^{-k\epsilon^2/8}
		\end{align*}
	\end{lemma}
	
	\vspace{8em}
	\begin{block}{\vspace*{-3ex}}
		\small \textbf{Goal}: Prove $\|\bs{\Pi} \bv{x} \|_2^2$ concentrates within $1 \pm \epsilon$ of its expectation, which equals $\|\bv{x} \|_2^2$.
	\end{block}
\end{frame}

\begin{frame}
	\frametitle{connection to earlier part of lecture}
	If high dimensional geometry is so different from low-dimensional geometry, why is \emph{dimensionality reduction possible?} 
	
	Doesn't Johnson-Lindenstrauss tell us that high-dimensional geometry can be approximated in low dimensions?
\end{frame}

\begin{frame}[t]
	\frametitle{connection to dimensionality reduction}
	\textbf{Hard case:} $\bv{x}_1, \ldots, \bv{x}_n \in \R^d$ are all mutually orthogonal unit vectors: 
	\begin{align*}
		\|\bv{x}_i - \bv{x}_j\|_2^2 &= 2 & &\text{for all $i,j$.}  
	\end{align*}
	When we reduce to $k$ dimensions with JL, we still expect these vectors to be nearly orthogonal. Why?
	
\end{frame}

\begin{frame}[t]
	\frametitle{connection to dimensionality reduction}
	\textbf{Hard case:} $\bv{x}_1, \ldots, \bv{x}_n \in \R^d$ are all mutually orthogonal unit vectors: 
	\begin{align*}
		\|\bv{x}_i - \bv{x}_j\|_2^2 &= 2 & &\text{for all $i,j$.}  
	\end{align*}
	
	From our result last class, in $O(\log n /\epsilon^2)$ dimensions, there exists $2^{O(\epsilon^2\cdot \log n /\epsilon^2)} \geq n $ unit vectors that are close to mutually orthogonal.
	\alert{$O(\log n /\epsilon^2)$ = \emph{just enough} dimensions.}
	\vspace{2em}
	
	% \textbf{Alternative view:} Without additional structure, we expect that learning/inference in $d$ dimensions requires $2^{O(d)}$ data points. If we really had a data set that large, then the JL bound would be vacous, since $\log(n) = O(d)$.
\end{frame}

% \begin{frame}
% 	\frametitle{euclidean dimensionality reduction}
% 	\begin{lemma}[Johnson-Lindenstrauss, 1984]
% 		For any set of $n$ data points $\bv{q}_1,\ldots, \bv{q}_n \in \R^d$ there exists a \emph{linear map} $\Pi: \R^d \rightarrow \R^k$ where $k = O\left(\frac{\log n}{\epsilon^2}\right)$ such that \emph{for all $i,j$},
% 		\begin{align*}
% 			(1-\epsilon)\|\bv{q}_i - \bv{q}_j\|_2 \leq \|\bs{\Pi}\bv{q}_i - \bs{\Pi}\bv{q}_j\|_2 \leq (1+\epsilon)\|\bv{q}_i - \bv{q}_j\|_2.
% 		\end{align*}
% 	\end{lemma}
% 	\begin{center}
% 		\includegraphics[height=.45\textheight]{jl_sketch.png}
% 	\end{center}
% \end{frame}

% \begin{frame}[t]
% 	\frametitle{second application}
% 	\textbf{k-means clustering}: For data set $\bv{a}_1, \ldots, \bv{a}_n$,  find clusters $C_1, \ldots, C_k \subseteq \{1, \ldots, n\}$ to minimize:
% 	\vspace{-2em}
	
% 	\begin{align*}
% 		Cost(C_1,\ldots, C_k) = \sum_{j=1}^k \frac{1}{2|C_j|}\sum_{u,v\in C_j} \|\bv{a}_u - \bv{a}_v\|_2^2.
% 	\end{align*}
	
% 	\vspace{-2em}
% 	\begin{center}
% 		\includegraphics[width=.5\textwidth]{kmeans4.png}
% 	\end{center}
% \end{frame}

% \begin{frame}[t]
% 	\frametitle{second application}
% 	\textbf{k-means clustering}: For data set $\bv{a}_1, \ldots, \bv{a}_n$,  find clusters $C_1, \ldots, C_k \subseteq \{1, \ldots, n\}$ to minimize:
% 	\vspace{-2em}
	
% 	\begin{align*}
% 		Cost(C_1,\ldots, C_k) = \sum_{j=1}^k \frac{1}{2|C_j|}\sum_{u,v\in C_j} \|\bv{a}_u - \bv{a}_v\|_2^2.
% 	\end{align*}
	
% 	\vspace{-2em}
% 	\begin{center}
% 		\vspace{-.5em}
% 		\includegraphics[width=.5\textwidth]{clustering_projected.png}
% 		\vspace{-.5em}
% 	\end{center}
% 	\textbf{Claim:} If I find the optimal clustering for $\bs{\Pi}\bv{a}_1, \ldots, \bs{\Pi}\bv{a}_n$ then its cost is less than $(1+\epsilon)$ times the cost of the best clustering obtained with the original data. 
% \end{frame}

\begin{frame}[t]
	\frametitle{second application}
	\textbf{k-means clustering}: Give data points $\bv{a}_1,\ldots, \bv{a}_n \in \R^d$, find centers $\bs{\mu}_1, \ldots, \bs{\mu}_k\in \R^d$ to minimize:
	\begin{align*}
		Cost(\bs{\mu}_1,\ldots, \bs{\mu}_k) = \sum_{i=1}^n \min_{j = 1,\ldots,k} \|\bs{\mu}_j - \bv{a}_i\|_2^2
	\end{align*}
	\begin{center}
		\includegraphics[width=.5\textwidth]{kmeans1.png}
	\end{center}
\end{frame}
\begin{frame}[t]
	\frametitle{sample application}
	\textbf{k-means clustering}: Give data points $\bv{a}_1,\ldots, \bv{a}_n \in \R^d$, find centers $\bs{\mu}_1, \ldots, \bs{\mu}_k\in \R^d$ to minimize:
	\begin{align*}
		Cost(\bs{\mu}_1,\ldots, \bs{\mu}_k) = \sum_{i=1}^n \min_{j = 1,\ldots,k} \|\bs{\mu}_j - \bv{a}_i\|_2^2
	\end{align*}
	\begin{center}
		\includegraphics[width=.5\textwidth]{kmeans2.png}
	\end{center}
\end{frame}
\begin{frame}[t]
	\frametitle{sample application}
	\textbf{k-means clustering}: Give data points $\bv{a}_1,\ldots, \bv{a}_n \in \R^d$, find centers $\bs{\mu}_1, \ldots, \bs{\mu}_k\in \R^d$ to minimize:
	\begin{align*}
		Cost(\bs{\mu}_1,\ldots, \bs{\mu}_k) = \sum_{i=1}^n \min_{j = 1,\ldots,k} \|\bs{\mu}_j - \bv{a}_i\|_2^2
	\end{align*}
	\begin{center}
		\includegraphics[width=.5\textwidth]{kmeans3.png}
	\end{center}
\end{frame}

\begin{frame}[t]
	\frametitle{k-means clustering}
	\textbf{Equivalent form}: Find clusters $C_1, \ldots, C_k \subseteq \{1, \ldots, n\}$ to minimize:
	\vspace{-3em}
	
	\begin{align*}
		Cost(C_1,\ldots, C_k) = \sum_{j=1}^k \frac{1}{2|C_j|}\sum_{u,v\in C_j} \|\bv{a}_u - \bv{a}_v\|_2^2.
	\end{align*}
	\vspace{-2em}
	\begin{center}
		\includegraphics[width=.5\textwidth]{kmeans4.png}
	\end{center}
	\textbf{Exercise:} Prove this to your self.
\end{frame}

\begin{frame}[t]
	\frametitle{k-means clustering}
	NP-hard to solve exactly, but there are many good approximation algorithms. All depend at least linearly on the dimension $d$. 
	
	\textbf{Approximation scheme}: Find clusters $\tilde{C}_1, \ldots, \tilde{C}_k$ for the $k = O\left(\frac{\log n}{\epsilon^2}\right)$ dimension data set $\bs{\Pi}\bv{a}_1, \ldots, \bs{\Pi}\bv{a}_n.$
	
	\vspace{-3em}
	\begin{center}
		\includegraphics[width=.6\textwidth]{clustering_projected.png}
	\end{center}
	\vspace{-2em}
	Argue these clusters are near optimal for $\bv{a}_1, \ldots, \bv{a}_n$.
\end{frame}




\begin{frame}[t]
	\frametitle{k-means clustering}
	\begin{align*}
		Cost(C_1,\ldots, C_k) &= \sum_{j=1}^k \frac{1}{2|C_j|}\sum_{u,v\in C_j} \|\bv{a}_u - \bv{a}_v\|_2^2 \\
		\widetilde{Cost}(C_1,\ldots, C_k) &= \sum_{j=1}^k \frac{1}{2|C_j|}\sum_{u,v\in C_j} \|\Pi\bv{a}_u - \Pi\bv{a}_v\|_2^2
	\end{align*}
		
		
		\textbf{Claim:} For any clusters $C_1, \ldots, C_k$:
		\begin{align*}
	(1-\epsilon) Cost(C_1, \ldots, C_k) \leq \widetilde{Cost}(C_1, \ldots, C_k)   \leq  (1+\epsilon) Cost(C_1, \ldots, C_k) 
		\end{align*}
	
\end{frame}

\begin{frame}[t]
	\frametitle{k-means clustering}
	Suppose we use an approximation algorithm to find clusters $B_1, \ldots, B_k$ such that:
	\begin{align*}
		\widetilde{Cost}(B_1,\ldots, B_k) \leq (1+\alpha)\widetilde{Cost}^*
	\end{align*}
	
	Then: 
	\begin{align*}
		{Cost}(B_1,\ldots, B_k) &\leq \frac{1}{1-\epsilon}\widetilde{Cost}(B_1,\ldots, B_k) \\
		&\leq(1+O(\epsilon)) (1+\alpha)\widetilde{Cost}^*\\
		&\leq (1+O(\epsilon))(1+\alpha)(1+\epsilon){Cost}^*\\
		&= \alert{\left(1 + O(\alpha + \epsilon)\right){Cost}^*}
	\end{align*}

	\vspace{1em}
\begin{block}{\vspace*{-3ex}}
	\small ${Cost}^* = \min_{C_1, \ldots, C_k} Cost(C_1, \ldots, C_k)$ and $\widetilde{Cost}^* = \min_{C_1, \ldots, C_k} \widetilde{Cost}(C_1, \ldots, C_k) $
\end{block}
\end{frame}


%\begin{frame}[standout]
%	\begin{center}
	%		break
	%	\end{center}
%\end{frame}
%
\begin{frame}
	\frametitle{dimensionality reduction}
	\begin{center}
		The Johnson-Lindenstrauss Lemma let us sketch vectors and preserve their \textbf{$\ell_2$ Euclidean distance.} 
		
		We also have dimensionality reduction techniques that preserve alternative measures of similarity.
	\end{center}
\end{frame}

\begin{frame}
	\frametitle{jaccard similarity}
	Often vector embeddings used in semantic search are \emph{binary}. For such vectors, \emph{Jaccard similarity} is often used instead of Euclidean distance or inner product to compute similarity.
	\begin{definition}[Jaccard Similarity]
			\begin{align*}
					J(\bv{q},\bv{y}) = \frac{|\bv{q} \cap \bv{y}|}{|\bv{q} \cup \bv{y}|} = \frac{\text{\# of non-zero entries in common}}{\text{total \# of non-zero entries}}
				\end{align*}
			Natural similarity measure for binary vectors. $0\leq J(\bv{q},\bv{y})\leq 1$.
		\end{definition}
\end{frame}


\begin{frame}
	\frametitle{jaccard similarity for document comparison}
	\textbf{``Bag-of-words'' model:}
	\begin{center}
			\includegraphics[width=.95\textwidth]{bagofwords.png}
		\end{center}
	
	How many words do a pair of documents have in common?
\end{frame}

\begin{frame}
	\frametitle{jaccard similarity for document comparison}
	\textbf{``Bag-of-words'' model:}
	\begin{center}
			\includegraphics[width=.95\textwidth]{bigrams.png}
		\end{center}
	
	How many bigrams do a pair of documents have in common?
\end{frame}

\begin{frame}
	\frametitle{applications: document similarity}
	\begin{itemize}
			\item Finding duplicate or new duplicate documents or webpages.
			\item Change detection for high-speed web caches.
			\item Finding near-duplicate emails or customer reviews which could indicate spam.
		\end{itemize}
\end{frame}

\begin{frame}
	\frametitle{jaccard similarity for seismic data}
	\begin{center}
			\vspace{-.5em}
			\includegraphics[width=.6\textwidth]{earthquakeFeatures.jpg}
			
			\vspace{-.5em}
			Feature extract pipeline for earthquake data.
			
			(see paper by Rong et al. posted on course website)
		\end{center}
\end{frame}




\begin{frame}
	\frametitle{similarity estimation}
	\textbf{Goal:} Design a compact sketch $C: \{0,1\}\rightarrow \R^k$:
	\begin{center}
			\vspace{-.5em}
			\includegraphics[width=.8\textwidth]{compression.png}
			\vspace{-.5em}
		\end{center}
	Want to use $C(\bv{q}), C(\bv{y})$ to approximately compute the Jaccard similarity $J(\bv{q},\bv{y}) = \frac{|\bv{q} \cap \bv{y}|}{|\bv{q} \cup \bv{y}|}$.
\end{frame}

\begin{frame}
	\frametitle{minhash}
	\textbf{MinHash (Broder, '97)}:
	\begin{itemize}
		\item Choose $k$ random hash functions $h_1, \ldots, h_k: \{1,\ldots, n\} \rightarrow [0,1]$. 
		\item For $i\in 1, \ldots,k$, 
		\begin{itemize}
			\item Let $c_i = \min_{j, \bv{q}_j = 1} h_i(j)$.
		\end{itemize}
		\item $C(\bv{q}) = [c_1, \ldots, c_k]$.
	\end{itemize}
	\begin{center}
		\includegraphics[width=\textwidth]{minHash1.png}	
	\end{center}
\end{frame}

\begin{frame}
	\frametitle{minhash}
	\begin{itemize}
		\item Choose $k$ random hash functions $h_1, \ldots, h_k: \{1,\ldots, n\} \rightarrow [0,1]$. 
		\item For $i\in 1, \ldots,k$, 
		\begin{itemize}
			\item Let $c_i = \min_{j, \bv{q}_j = 1} h_i(j)$.
		\end{itemize}
		\item $C(\bv{q}) = [c_1, \ldots, c_k]$.
	\end{itemize}
	\begin{center}
		\includegraphics[width=\textwidth]{minHash2.png}	
	\end{center}
\end{frame}

\begin{frame}[t]
	\frametitle{minhash analysis}
	\textbf{Claim:} For all $i$, $\Pr[c_i(\bv{q}) = c_i(\bv{y})] = J(\bv{q},\bv{y}) = \frac{|\bv{q} \cap \bv{y}|}{|\bv{q} \cup \bv{y}|}$.
	\begin{center}
		\includegraphics[width=.8\textwidth]{minHashSimple.png}
	\end{center}
	\textbf{Proof:} 
	
	1. For $c_i(\bv{q}) = c_i(\bv{y})$, we need that $\argmin_{i\in \bv{q}} h(i) = \argmin_{i\in \bv{y}} h(i)$.
\end{frame}

\begin{frame}[t]
	\frametitle{minhash analysis}
	\textbf{Claim:} $\Pr[c_i(\bv{q}) = c_i(\bv{y})] = J(\bv{q},\bv{y})$.	
	\begin{center}
		\includegraphics[width=.8\textwidth]{minhash_colored.png}
	\end{center}
	2. Every non-zero index in $\bv{q}\cup \bv{y}$ is equally likely to produce the lowest hash value.
	$c_i(\bv{q}) = c_i(\bv{y})$ only if this index is 1 in \emph{both} $\bv{q}$ and $\bv{y}$. There are $\bv{q}\cap \bv{y}$ such indices. So:
	\begin{align*}
		\Pr[c_i(\bv{q}) =c_i(\bv{y})]= \frac{|\bv{q}\cap \bv{y}|}{|\bv{q}\cup \bv{y}|}  = J(\bv{q},\bv{y})
	\end{align*}
\end{frame}


\begin{frame}
	\frametitle{minhash analysis}
	Let $J = J(\bv{q},\bv{y})$ denote the Jaccard similarity between $\bv{q}$ and $\bv{y}$. \vspace{1em}
	
	\textbf{Return:} $\tilde{J} = \frac{1}{k} \sum_{i=1}^k \mathbbm{1}[c_i(\bv{q}) = c_i(\bv{y})]$. 
	
	\textbf{Unbiased estimate for Jaccard similarity:}
	\begin{align*}
		\E \tilde{J} = \hspace{14em}
	\end{align*}
	
	\begin{center}
		\includegraphics[width=.6\textwidth]{minHashCompare.png}
	\end{center}
	The more repetitions, the lower the variance. 
\end{frame}

\begin{frame}[t]
	\frametitle{minhash analysis}
	Let $J = J(\bv{q},\bv{y})$ denote the true Jaccard similarity.
	
	\textbf{Estimator:} $\tilde{J} = \frac{1}{k} \sum_{i=1}^k \mathbbm{1}[c_i(\bv{q}) = c_i(\bv{y})]$. 
	\begin{align*}
		\Var [\tilde{J}] =\hspace{16em}
	\end{align*}
	
	Plug into Chebyshev inequality. How large does $k$ need to be so that with probability $> 1 - \delta$, $|J-\tilde{J}| \leq \epsilon$?
\end{frame}

\begin{frame}
	\frametitle{minhash analysis}
	\textbf{Chebyshev inequality:} As long as $\alert{{k = O\left(\frac{1}{\epsilon^2\delta}\right)}}$, then with prob. $1-\delta$,
	\begin{align*}
		J(\bv{q}, \bv{y}) -\epsilon \leq \tilde{J}\left(C(\bv{q}),C(\bv{y})\right)   \leq J(\bv{q}, \bv{y}) + \epsilon. 
	\end{align*}
	\begin{center}
		And $\tilde{J}$ only takes $O(k)$ time to compute! \alert{\textbf{Independent}} of original vector dimension, $d$.
	\end{center}	
	
	Can be improved to $\log(1/\delta)$ dependence?
\end{frame}



\begin{frame}
	\frametitle{vector search / near neighbor search}
	\textbf{Goal:} Find all vectors in database $\bv{q}_1, \ldots, \bv{q}_n \in \R^d$ that are close to some input query vector $\bv{y}\in \R^d$. I.e. find all of $\bv{y}$'s ``nearest neighbors'' in the database.

		\begin{center}
	\textbf{\alert{How does similarity sketching help in these applications?}}
		\end{center}
	\begin{itemize}
		\item Improves runtime of ``linear scan'' from $O(nd)$ to $O(nk)$.
		\item Improves space complexity from $O(nd)$ to $O(nk)$. This can be super important -- e.g. if it means the linear scan only accesses vectors in fast memory.
	\end{itemize}

	\begin{center}
		\textbf{Can we also reduce the dependence on $n$?}
	\end{center}
\end{frame}

\begin{frame}
	\frametitle{beyond a linear scan}
	\begin{center}
		\textbf{Goal:} \emph{Sublinear} $o(n)$ time to find near neighbors. 
	\end{center}
\end{frame}

\begin{frame}
	\frametitle{beyond a linear scan}	
	This problem can already be solved in low-dimensions using space partitioning approaches (e.g. kd-tree).
	
	\includegraphics[height=.4\textheight]{kdtree.png}	\includegraphics[height=.4\textheight]{3dtree.png}
	
	Runtime is roughly $O(d\cdot \min(n,2^d))$, which is only sublinear for $d = o(\log n)$.

\end{frame}

\begin{frame}
	\frametitle{high dimensional near neighbor search}	
	Only been attacked much more recently:
	\begin{itemize}
		\item \textbf{\alert{Locality-sensitive hashing [Indyk, Motwani, 1998]}}
		\item Spectral hashing [Weiss, Torralba, and Fergus, 2008]
		\item Vector quantization [J\'{e}gou, Douze, Schmid, 2009]
		\item Graph-based vector search [Malkov, Yashunin, 2016, Subramanya et al., 2019]
	\end{itemize}

\textbf{Key ideas behind all of these methods:} 
\begin{enumerate}
\item Trade worse space-complexity + preprocessing time for better time-complexity. I.e., preprocess database in data structure that uses $\Omega(n)$ space.
\item Allow for approximation.
\end{enumerate}
\end{frame}

\begin{frame}
	\frametitle{locality sensitive hash functions}
	Let $h: \R^d \rightarrow \{1, \ldots, m\}$ be a random hash function. 
	
	We call $h$ \emph{locality sensitive} for similarity function $s(\bv{q},\bv{y})$ if $\Pr\left[h(\bv{q}) == h(\bv{y})\right]$ is:
	\begin{itemize}
		\item Higher when $\bv{q}$ and $\bv{y}$ are more similar, i.e. $s(\bv{q},\bv{y})$ is higher.
		\item Lower when $\bv{q}$ and $\bv{y}$ are more dissimilar, i.e. $s(\bv{q},\bv{y})$ is lower. 
	\end{itemize}
\begin{center}
	\includegraphics[width=.9\textwidth]{cam_lsh.png}
\end{center}
\end{frame}

\begin{frame}
	\frametitle{locality sensitive hash functions}
	LSH for $s(\bv{q},\bv{y})$  equal to Jaccard similarity:
	\begin{itemize}
		\item Let $c: \{0,1\}^d \rightarrow [0,1]$ be a single instantiation of MinHash.
		\item Let $g: [0,1] \rightarrow \{1, \ldots, m\}$ be a uniform random hash function.
		\item Let $h(\bv{q}) = g(c(\bv{q})).$
	\end{itemize}
\begin{center}
	\includegraphics[width=.7\textwidth]{single_min_hash.png}
\end{center}

\end{frame}

\begin{frame}
	\frametitle{locality sensitive hash functions}
	LSH for Jaccard similarity:
	\begin{itemize}
		\item Let $c: \{0,1\}^d \rightarrow [0,1]$ be a single instantiation of MinHash.
		\item Let $g: [0,1] \rightarrow \{1, \ldots, m\}$ be a uniform random hash function.
		\item Let $h(\bv{x}) = g(c(\bv{x})).$
	\end{itemize}
		If $J(\bv{q},\bv{y}) = v$, 
		\begin{align*}
			\Pr\left[h(\bv{q}) == h(\bv{y})\right] = \hspace{15em}
		\end{align*}
\end{frame}

\begin{frame}
	\frametitle{near neighbor search}
	\begin{center}
	Basic approach for LSH-based near neighbor search in a database.
	\end{center}
		\textbf{Pre-processing:}
	\begin{itemize}
		\item Select random LSH function $h: \{0,1\}^d \rightarrow 1,\ldots, m$.
		\item Create table $T$ with $m = O(n)$ slots.\footnote{Enough to make the $O(1/m)$ term negligible.}
		\item For $i = 1,\ldots, n$, insert $\bv{q}_i$ into $T(h(\bv{q}_i))$.
	\end{itemize}
	\textbf{Query:}
	\begin{itemize}
		\item Want to find near neighbors of input $\bv{y}\in\{0,1\}^d$.
		\item Linear scan through all vectors $\bv{q}\in T(h(\bv{y}))$ and return any that are close to $\bv{y}$. Time required is $O(d\cdot |T(h(\bv{y})|)$.
	\end{itemize}
\vspace{1em}
\end{frame}

\begin{frame}
	\frametitle{near neighbor search}
	\begin{center}
		\includegraphics[width=.8\textwidth]{basicScheme.png}
	\end{center}
\end{frame}

\begin{frame}
	\frametitle{near neighbor search}
	\textbf{Two main considerations:}
	\begin{itemize}
		\item \textbf{False Negative Rate}: What's the probability we do not find a vector that \emph{is close} to $\bv{y}$?
		\item \textbf{False Positive Rate}: What's the probability that a vector in $T(h(\bv{y}))$ \emph{is not close} to $\bv{y}$?
	\end{itemize}

A higher false negative rate means we miss near neighbors.

A higher false positive rate means increased runtime -- we need to compute $S(\bv{q},\bv{y})$ for every $\bv{q}\in T(h(\bv{y}))$ to check if it's actually close to $\bv{y}$.

\textbf{Note:} The meaning of ``close'' and ``not close'' is application dependent. E.g. we might specify that we want to find anything with Jaccard similarity $> .4$, but not with Jaccard similarity $< .2$. 
\end{frame}

\begin{frame}[t]
	\frametitle{reducing false negative rate}	
	Let's use Jaccard similarity as a running example. We will discuss LSH for inner product/Euclidean distance as well. \\

	Suppose the nearest database point $\bv{q}$ has $J(\bv{y},\bv{q}) = .4$.
	\begin{center}
	\textbf{\alert{What's the probability we do not find $\bv{q}$?}}
	\end{center}
\end{frame}

\begin{frame}
	\frametitle{reducing false negative rate}
	\begin{center}
		\includegraphics[width=.8\textwidth]{many_tables.png}
	\end{center}
	\textbf{Pre-processing:}
	\begin{itemize}
		\item Select $t$ independent LSH's $h_1, \ldots, h_t: \{0,1\}^d \rightarrow 1,\ldots, m$.
		\item Create tables $T_1, \ldots, T_t$, each with $m$ slots. 
		\item For $i = 1,\ldots, n$, $j = 1,\ldots, t$, 
		\begin{itemize}
			\item Insert $\bv{q}_i$ into $T_j(h_j(\bv{q}_i))$.
		\end{itemize}
	\end{itemize}
\end{frame}

\begin{frame}[t]
	\frametitle{reducing false negative rate}
	\textbf{Query:}
	\begin{itemize}
		\item Want to find near neighbors of input $\bv{y}\in\{0,1\}^d$.
		\item Linear scan through all vectors in $T_1(h_1(\bv{y}))\cup T_2(h_2(\bv{y}))\cup \ldots, T_t(h_t(\bv{y}))$.
	\end{itemize}

\vspace{2em}
	Suppose the nearest database point $\bv{q}$ has $J(\bv{y},\bv{q}) = .4$.
	\begin{center}
		\textbf{\alert{What's the probability we find $\bv{q}$?}}
	\end{center}
	
	\vskip0pt plus 1filll
	\hspace{-1em}\scriptsize($10$, $99\%$)
\end{frame}

\begin{frame}[t]
	\frametitle{what happens to false positives?}
	Suppose there is some other database point $\bv{z}$ with $J(\bv{y},\bv{z}) = .2$. 
	
	What is the probability we will need to compute $J(\bv{z},\bv{y})$ in our hashing scheme with one table? I.e. the probability that $\bv{y}$ hashes into at least one bucket containing $\bv{z}$. 
	
	\vspace{4em}
	{\textbf{\alert{In the new scheme with $t=10$ tables?}}}
	
	\vskip0pt plus 1filll
	\hspace{-1em}\scriptsize($89\%$)
\end{frame}

\begin{frame}[t]
	\frametitle{reducing false positives}
	\small
	\vspace{-.5em}
	\begin{center}
	\textbf{Change our locality sensitive hash function}.
	\vspace{-.5em}
	\end{center}
	\emph{Tunable} LSH for Jaccard similarity:
	\vspace{-.5em}
	\begin{itemize}
		\item Choose parameter $r \in \mathbb{Z}^+$.
		\item Let $c_1, \ldots, c_r: \{0,1\}^d \rightarrow [0,1]$ be independnt random MinHash's.
		\item Let $g: [0,1]^r \rightarrow \{1, \ldots, m\}$ be a uniform random hash function.
		\item Let $h(\bv{x}) = g(c_1(\bv{x}), \ldots, c_r(\bv{x})).$
	\end{itemize}
\vspace{-1em}
	\begin{center}
			\includegraphics[width=.8\textwidth]{banded_hash.png}
	\end{center}


\end{frame}

\begin{frame}[t]
	\frametitle{reducing false positives}
	\small
	\emph{Tunable} LSH for Jaccard similarity:
	\begin{itemize}
		\item Choose parameter $r \in \mathbb{Z}^+$.
		\item Let $c_1, \ldots, c_r: \{0,1\}^d \rightarrow [0,1]$ be random MinHash.
		\item Let $g: [0,1]^r \rightarrow \{1, \ldots, m\}$ be a uniform random hash function.
		\item Let $h(\bv{x}) = g(c_1(\bv{x}), \ldots, c_r(\bv{x})).$
	\end{itemize}
	
	If $J(\bv{q},\bv{y}) = v$, then $\Pr\left[h(\bv{q}) == h(\bv{y})\right] = $
	
\end{frame}

\begin{frame}[t]
	\frametitle{tunable lsh}
	\begin{center}
		\includegraphics[width=.8\textwidth]{tuning_minhash.png}
	\end{center}
\end{frame}

\begin{frame}[t]
	\frametitle{tunable lsh}
	Full LSH cheme has two parameters to tune:
	\begin{center}
		\includegraphics[width=.8\textwidth]{full_scheme.png}
	\end{center}
\end{frame}

\begin{frame}[t]
	\frametitle{tunable lsh}
	Effect of \textbf{increasing number of tables} $t$ on:
	\begin{center}
		False Negatives \hspace{6em} False Positives
	\end{center}
\vspace{4em}
	Effect of \textbf{increasing number of bands} $r$ on:
\begin{center}
	False Negatives \hspace{6em} False Positives
\end{center}
\end{frame}

%\begin{frame}
%	\frametitle{some examples}
%	Choose tables $t$ large enough so false negative rate to $1\%$.
%	\begin{center}
%		\alert{\textbf{Parameter:} $\mathbf{r = 1}$.}
%	\end{center}
%	Chance we find $\bv{q}$ with $J(\bv{y},\bv{q}) = .8$:
%	\vspace{5em}
%	
%	Chance we need to check $\bv{z}$ with $J(\bv{y},\bv{z}) = .4$:
%\end{frame}
%
%\begin{frame}
%	\frametitle{some examples}
%	Choose tables $t$ large enough so false negative rate to $1\%$.
%	\begin{center}
%		\alert{\textbf{Parameter:} $\mathbf{r = 2}$.}
%	\end{center}
%	Chance we find $\bv{q}$ with $J(\bv{y},\bv{q}) = .8$:
%	\vspace{5em}
%	
%	Chance we need to check $\bv{z}$ with $J(\bv{y},\bv{z}) = .4$:
%\end{frame}
%
%\begin{frame}
%	\frametitle{some examples}
%	Choose tables $t$ large enough so false negative rate to $1\%$.
%	\begin{center}
%		\alert{\textbf{Parameter:} $\mathbf{r = 5}$.}
%	\end{center}
%	Chance we find $\bv{q}$ with $J(\bv{y},\bv{q}) = .8$:
%	\vspace{5em}
%	
%	Chance we need to check $\bv{z}$ with $J(\bv{y},\bv{z}) = .4$:
%\end{frame}

\begin{frame}
	\frametitle{$s$-curve tuning}
	Probability we check $\bv{q}$ when querying $\bv{y}$ if $J(\bv{q},\bv{y}) = v$:
	\begin{align*}
%	\approx 	1 - (1 - v^r)^t
	\end{align*}
	\begin{center}
		\includegraphics[width=.6\textwidth]{scurve_5_5.png}
		
		$r = 5, t = 5$
	\end{center}
\end{frame}

\begin{frame}
	\frametitle{$s$-curve tuning}
	Probability we check $\bv{q}$ when querying $\bv{y}$ if $J(\bv{q},\bv{y}) = v$:
	\begin{align*}
	\approx 1 - (1 - v^r)^t
	\end{align*}
	\begin{center}
		\includegraphics[width=.6\textwidth]{scurve_5_40.png}
		
		$r = 5, t = 40$
	\end{center}
\end{frame}

\begin{frame}
	\frametitle{$s$-curve tuning}
	Probability we check $\bv{q}$ when querying $\bv{y}$ if $J(\bv{q},\bv{y}) = v$:
	\begin{align*}
	\approx 1 - (1 - v^r)^t 
	\end{align*}
	\begin{center}
		\includegraphics[width=.6\textwidth]{scurve_40_5.png}
		
		$r = 40, t = 5$
	\end{center}
\end{frame}

\begin{frame}
	\frametitle{$s$-curve tuning}
	Probability we check $\bv{q}$ when querying $\bv{y}$ if $J(\bv{q},\bv{y}) = v$:
	\begin{align*}
	1 - (1 - v^r)^t
	\end{align*}
	\begin{center}
		\includegraphics[width=.6\textwidth]{scurve_centered.png}
		
		Increasing both $r$ and $t$ gives a steeper curve. 
		
		\alert{\textbf{Better for search, but worse space complexity}.}
	\end{center}
\end{frame}

\begin{frame}
	\frametitle{fixed threshold}
	\small
	\textbf{Use Case 1:} Fixed threshold.
	\begin{itemize}
		\item Shazam wants to find match to audio clip $\bv{y}$ in a database of 10 million  clips.
		\item There are 10 \emph{true matches} with $J(\bv{y},\bv{q}) > .9$.
		\item There are 10,000 \emph{near matches} with $J(\bv{y},\bv{q}) \in [.7,.9]$.
		\item All other items have $J(\bv{y},\bv{q}) < .7$.
	\end{itemize}
	With $r = 25$ and $t = 40$, 
	\begin{itemize}
		\item Hit probability for $J(\bv{y},\bv{q}) > .9$ is $\gtrsim 1 - (1 - .9^{25})^{40} = .95$
		\item Hit probability for $J(\bv{y},\bv{q}) \in [.7,.9]$ is $\lesssim 1 - (1 - .9^{25})^{40} = .95$
		\item Hit probability for $J(\bv{y},\bv{q}) < .7$ is $\lesssim 1 - (1 - .7^{25})^{40} = .005$
	\end{itemize}
	\textbf{Upper bound on total number of items checked:} 
	\begin{align*}
	10 + .95 \cdot 10,000 + .005 \cdot 9,989,990 \alert{\approx 60,000 \ll 10,000,000}.
	\end{align*}  
\end{frame}

\begin{frame}
	\frametitle{fixed threshold}
	\begin{center}
	Space complexity: 40 hash tables \alert{$\approx 40\cdot O(n)$}. 
	
	\textbf{Directly trade space for fast search.}
	\end{center}
\end{frame}

\begin{frame}
	\frametitle{worse case guarantees}
	\begin{center}
	\textbf{Near Neighbor Search Problem}
\end{center}

	Concrete worst case result:
	\begin{theorem}[Indyk, Motwani, 1998]
		If there exists some $q$ with $\|\bv{q} - \bv{y}\|_0 \leq R$, return a vector $\tilde{\bv{q}}$ with $\|\tilde{\bv{q}} - \bv{y}\|_0 \leq C\cdot R$ in:
		\begin{itemize}
			\item Time: $O\left(n^{1/C}\right)$.
			\item Space: $O\left(n^{1 + 1/C}\right)$. 
		\end{itemize}
	\end{theorem}
	$\|\bv{q} - \bv{y}\|_0 = $ ``hamming distance" = number of elements that differ between $\bv{q}$ and $\bv{y}$. 
\end{frame}

\begin{frame}
	\frametitle{approximate nearest neighbor search}
		\begin{theorem}[Indyk, Motwani, 1998]
		Let $q$ be the closest database vector to $\bv{y}$. Return a vector $\tilde{\bv{q}}$ with $\|\tilde{\bv{q}} - \bv{y}\|_0 \leq C\cdot \|{\bv{q}} - \bv{y}\|_0$ in:
		\begin{itemize}
			\item Time: $\tilde{O}\left(n^{1/C}\right)$.
			\item Space: $\tilde{O}\left(n^{1 + 1/C}\right)$. 
		\end{itemize}
	\end{theorem}
	Similar results can be proven for other metrics, including Euclidean distance. But you need a good LSH function.
\end{frame}

\begin{frame}
	\frametitle{other lsh functions}
	\begin{center}
	Good locality sensitive hash functions exists for other similarity measures.
	\end{center}
	\textbf{Cosine similarity $\cos\left(\theta(\bv{x},\bv{y})\right) = \frac{\langle \bv{x},\bv{y}\rangle}{\|\bv{x}\|_2\|\bv{y}\|_2}$:}
	\begin{center}
		\includegraphics[width=.7\textwidth]{cos_sim.png}
		
		$-1 \leq \cos\left(\theta(\bv{x},\bv{y})\right) \leq 1$.
	\end{center}
\end{frame}

\begin{frame}
	\frametitle{cosine similarity}
	\begin{center}
		Cosine similarity is natural ``inverse" for Euclidean distance.
	\end{center}
		\textbf{Euclidean distance $\|\bv{x} - \bv{y}\|_2^2$:}
		\begin{itemize}
			\item Suppose for simplicity that $\|\bv{x}\|_2^2 = \|\bv{y}\|_2^2 = 1$.
		\end{itemize}
\end{frame}

\begin{frame}
	\frametitle{simhash}
	Locality sensitive hash for \textbf{cosine similarity}:
	\begin{itemize}
		\item Let $\bv{g} \in \R^d$ be randomly chosen with each entry $\mathcal{N}(0,1)$. 
		\item Let $f: \{-1,1\} \rightarrow \{1,\ldots, m\}$ be a uniformly random hash function. 
		\item $h: \R^d \rightarrow \{1,\ldots, m\}$ is definied $h(\bv{x}) = f\left(\sign(\langle \bv{g}, \bv{x} \rangle)\right)$.
	\end{itemize}
	\begin{center}
		\alert{\textbf{
				\large
				If $\cos(\theta(\bv{x},\bv{y})) = v$, what is $\Pr[h(\bv{x}) == h(\bv{y})]$?
		}}
	\end{center}
\end{frame}


%\begin{frame}
%	\frametitle{simhash}
%	\begin{center}
%		\textbf{Inspired by Johnson-Lindenstrauss sketching}
%		
%		\includegraphics[width=\textwidth]{simhash_jl.png}
%	\end{center}
%\end{frame}



\begin{frame}[t]
	\frametitle{simhash analysis in 2d}
	\textbf{Theorem (to be proven):} If $\cos(\theta(\bv{x},\bv{y})) = v$, then 
	\begin{align*}
		\Pr[h(\bv{x}) == h(\bv{y})] = 1 - \frac{\theta}{\pi}  + \frac{\theta/\pi}{m}= 1 - \frac{\cos^{-1}(v)}{\pi} + \frac{\theta/\pi}{m}
	\end{align*}
	\begin{center}
		\includegraphics[width=.6\textwidth]{cosine_sim.png}
	\end{center}
\end{frame}

\begin{frame}
	\frametitle{simhash}
	SimHash can be banded, just like our MinHash based LSH function for Jaccard similarity:
	\begin{itemize}
		\item Let $\bv{g}_1,\ldots, \bv{g}_r \in \R^d$ be randomly chosen with each entry $\mathcal{N}(0,1)$. 
		\item Let $f: \{-1,1\}^r \rightarrow \{1,\ldots, m\}$ be a uniformly random hash function. 
		\item $h: \R^d \rightarrow \{1,\ldots, m\}$ is defined $h(\bv{x}) = f\left([\sign(\langle \bv{g}_1, \bv{x} \rangle),\ldots, \sign(\langle \bv{g}_r, \bv{x} \rangle)]\right)$.
	\end{itemize}
\begin{align*}
	\Pr[h(\bv{x}) == h(\bv{y})] \approx \left(1-\frac{\theta}{\Pi}\right)^r
\end{align*}
\end{frame}

\begin{frame}
	\textbf{To prove:}  $\Pr[h(\bv{x}) == h(\bv{y})] \approx 1 - \frac{\theta}{\pi}$,  where $h(\bv{x}) = f\left(\sign(\langle \bv{g}, \bv{x} \rangle)\right)$ and $f$ is uniformly random hash function.
	\frametitle{simhash analysis in 2d}
	\vspace{-.5em}
	\begin{center}
		\includegraphics[width=.5\textwidth]{simhash1.png}
	\end{center}
\begin{align*}
	\Pr[h(\bv{x}) == h(\bv{y})] = z + \frac{1-z}{m} \approx z.
\end{align*}
where $z = \Pr[\sign(\langle \bv{g}, \bv{x} \rangle) == \sign(\langle \bv{g}, \bv{y} \rangle)]$
\end{frame}

\begin{frame}
	\frametitle{simhash analysis 2d}
	\vspace{-.5em}
	\begin{center}
		\includegraphics[width=.55\textwidth]{simhash2.png}
	\end{center}
\vspace{-.5em}
$\Pr[h(\bv{x}) == h(\bv{y})] \approx$ probability $\bv{x}$ and $\bv{y}$ are on the same side of hyperplane orthogonal to $\bv{g}$.
\end{frame}

\begin{frame}
	\frametitle{simhash analysis higher dimensions}
	\begin{center}
		\includegraphics[width=.7\textwidth]{high_dim1.png}
	\end{center}
There is always some \emph{rotation matrix} $\bv{U}$ such that $\bv{U}\bv{x},\bv{U}\bv{y}$ are spanned by the first two-standard basis vectors and have the same cosine similarity as $\bv{x}$ and $\bv{y}$.
\end{frame}

\begin{frame}
	\frametitle{simhash analysis higher dimensions}
	\begin{center}
		\includegraphics[width=.7\textwidth]{high_dim2.png}
	\end{center}
	There is always some \emph{rotation matrix} $\bv{U}$ such that $\bv{x},\bv{y}$ are spanned by the first two-standard basis vectors. 
	
	
	\textbf{Note:} A rotation matrix $\bv{U}$ has the property that $\bv{U}^T\bv{U} = \bv{I}$. I.e., $\bv{U}^T$ is a rotation matrix itself, which reverses the rotation of $\bv{U}$.
\end{frame}

\begin{frame}[t]
	\frametitle{simhash analysis higher dimensions}
\textbf{Claim:} 
% \begin{align*}
% 1 - \frac{\theta}{\pi} &= \Pr[\sign(\langle \bv{g}[1,2], (\bv{U}\bv{x})[1,2] \rangle) == \sign(\langle \bv{g}[1,2], (\bv{U}\bv{y}[1,2] \rangle)] \\
% &= \Pr[\sign(\langle \bv{g}, \bv{U}\bv{x} \rangle) == \sign(\langle \bv{g}, \bv{U}\bv{y} \rangle)] \\
% &\alert{= \Pr[\sign(\langle \bv{g}, \bv{x} \rangle) == \sign(\langle \bv{g}, \bv{y} \rangle)]}
% \end{align*}
\begin{align*}
	&\Pr[\sign(\langle \bv{g}, \bv{x} \rangle) == \sign(\langle \bv{g}, \bv{y} \rangle) = \Pr[\sign(\langle \bv{g}, \bv{U}\bv{x} \rangle) == \sign(\langle \bv{g}, \bv{U}\bv{y} \rangle)] \\
	&\hspace{4em}= \Pr[\sign(\langle \bv{g}[1,2], (\bv{U}\bv{x})[1,2] \rangle) == \sign(\langle \bv{g}[1,2], (\bv{U}\bv{y}[1,2] \rangle)]\\
	&\hspace{4em} = 1 - \frac{\theta}{\pi}.
	\end{align*}

	The first step is the trickiest here. Why does it hold?
\end{frame}

% \begin{frame}[t]
% 	\frametitle{modern near neigbhor search}
% 	\begin{itemize}
% 		\item High-dimensional vector search is exploding as a research area with the rise of machine-learned multi-modal embeddings for images, text, and more. 
% 	\end{itemize}
% 	\begin{center}
% 		\vspace{-.5em}
% 				\includegraphics[width=.8\textwidth]{multimodal_embeddings.png}
% 				\vspace{-.5em}
% 	\end{center}
% 	Web-scale image search is now a vector search problem. 
% \end{frame}

% \begin{frame}[t]
% 	\frametitle{graph based near neigbhor}
	
% \end{frame}


\end{document} 








